%%%%%%%%%%%%%%%%%%%%%%%%%%%%%%%%%%%%%%%%%%%%%%%%%%%%%%%%%%%%%%%%%%%%%%%%%%%%%%%%
\section{Related Work}
%%%%%%%%%%%%%%%%%%%%%%%%%%%%%%%%%%%%%%%%%%%%%%%%%%%%%%%%%%%%%%%%%%%%%%%%%%%%%%%%

Systematic exploration is a relatively young approach to concurrency verification.
Recent work has focused on theoretical aspects of the technique, on distributed systems, and in userspace in general.
Other efforts have focused on kernel verification apart from systematic exploration, using techniques such as data race detection and formal model verification.

Landslide distinguishes itself from related systematic testing tools in its field as more of an exploratory work, presenting techniques to make systematic testing compatible with kernel-level code; less so as a development on the core testing approach itself.

\subsection{Systematic Exploration}

Dynamic Partial Order Reduction (DPOR) \cite{dpor,sdpor,dbug-retreat} is an algorithm for reducing the state space of ``all possible thread interleavings'' by identifying independent thread transitions and pruning the redundant state spaces that result from interleaving those transitions. Landslide makes generous use of the DPOR algorithm.

dBug\cite{dbug-ssv} is a systematic testing tool which makes use of DPOR. It focuses on multithreaded userspace applications, identifying when to preempt by interposing on \texttt{libc} library calls. It uses a message-passing model for inter-thread communication and for establishing a happens-before relationship between transitions. The dBug project heavily inspired the development of Landslide.

CHESS\cite{chess} is another userspace systematic testing tool. They explore the insight that many races require very few forced preemptions to uncover, and develop a search strategy which prioritises thread interleavings with fewer preemptions. Landslide does not (yet) make use of this strategy, though we did find a related insight which we discuss in Section~\ref{sec:future-backwards}.

There are also several tools, such as MaceMC\cite{macemc} and MODIST\cite{modist}, which provide systematic testing frameworks for networked and distributed applications.

DeMeter\cite{demeter} is a more recent tool for systematic exploration that introduces Dynamic Interface Reduction as a strategy for constraining the size of the state space, which means targetting the testing technique on only one component of the program at a time. Landslide makes use of the same general concept in its user interface (Section~\ref{sec:using}) - it provides the user with options to constrain decision point identification to certain subsets of the kernel.

\subsection{Kernel-level Verification}

% TODO

\subsection{Orthogonal Testing Techniques}

Research in dynamic verification has also seen other techniques apart from systematic exploration.

% TODO

% vim: ft=tex
