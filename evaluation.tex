%%%%%%%%%%%%%%%%%%%%%%%%%%%%%%%%%%%%%%%%%%%%%%%%%%%%%%%%%%%%%%%%%%%%%%%%%%%%%%%%
\section{Evaluation}
%%%%%%%%%%%%%%%%%%%%%%%%%%%%%%%%%%%%%%%%%%%%%%%%%%%%%%%%%%%%%%%%%%%%%%%%%%%%%%%%
% 410 - talk about current state/methods (for students and for TAS)
% test suite presented
% process of instrumenting

\subsection{User Experience}

%\subsubsection{

\subsection{Bug Case Studies}
\label{sec:eval-casestudy}

% TODO: intro text goes here

\subsubsection{Numbers}

Table~\ref{fig:numbers} shows the time it takes to find each of these bugs using Landslide, configured with several different sets of decision points, and using conventional stress testing.
All numbers represent the average of 5 trials. ``N/A ($X$)'' indicates the bug was not found with a given test setup (and instead, Landslide took $X$ time to explore the entire decision tree). ``$infty$'' indicates the bug was not found with stress testing. % TODO: after how long? what timeout?
All Landslide trial times include the Simics start-up and kernel boot-up time (time between issuing the command and the test case beginning to run), roughly 15 seconds.

All Landslide trials were run on the Gates-Hillman cluster machines (2.6 GHz Xeon).
% TODO: say specs of crash machine

\begin{figure*}[t!]
	\begin{tabular}{|l|l||c|c|c|c||c|}
		\hline
		\multirow{2}{*}{\bf Kernel and test case} & \multirow{2}{*}{\bf Bug description} & \multicolumn{4}{c||}{\bf Custom decision points} & \multirow{2}{*}{\bf Stress test} \\
		\cline{3-6}
		& & \bf default & \bf lock & \bf unlock & \bf both & \\
		\hline\hline
		POBBLES vanish/vanish (a) & Deadlock & & & & & \\
		\hline
		POBBLES vanish/vanish (b) & Data race (panic) & & & & & \\
		\hline
		POBBLES wait/wait & Data structure bug (panic) & & 38.6s & & & \\
		\hline
		POBBLES thread\_fork/vanish & Use-after-free & & & & & \\
		\hline
		LudicrOS vanish/vanish & Data race (infinite loop) & & & & & \\
		\hline
		LudicrOS yield/vanish & Use-after-free & & & & & \\
		\hline
	\end{tabular}
	\caption{Comparison of time taken to find bugs using Landslide and using conventional stress testing.
	Landslide's times are given for several different sets of decision points: the default set, consisting only of voluntary reschedules (Section~\ref{sec:components-arbiter}); and using custom decision points in addition to the default set: calls to \texttt{mutex\_lock}, calls to \texttt{mutex\_unlock}, and both.}
	\label{fig:numbers}
\end{figure*}

All Landslide trials were run with ``backwards exploration'' enabled (Section~\ref{sec:using-search}).
All trials were also run with Landslide configured to pay attention to only the relevant system calls (using \texttt{within\_function}; Section~\ref{sec:using-decision}).

We believe it is reasonable to test for these bugs in this way - using the minimal set of system calls to be paid attention to as necessary to find the bug - because it follows the recommended workflow of using Landslide, which is to start with what the user judges to be the ``smallest relevant set'' of decision points. The configuration using \texttt{within\_function} was as follows.

\begin{itemize}
	\item POBBLES vanish/vanish(a): \texttt{within\_function vanish}.
	\item POBBLES vanish/vanish(b): \texttt{within\_function vanish}.
	\item POBBLES wait/wait: \texttt{within\_function wait}.
	\item POBBLES thread\_fork/vanish: none was needed, but \texttt{within\_function thread\_fork} and \texttt{within\_function vanish} both were used for the sake of demonstration.
	\item LudicrOS vanish/vanish: \texttt{within\_function vanish}.
	\item LudicrOS yield/vanish: \texttt{within\_function yield}.
\end{itemize}

%%%%

Table~\ref{fig:trees} presents more detailed information about the decision trees that Landslide explored when finding these bugs
For each set of decision points on each bug, we give the total number of decision points in the tree, the total number of backtracks (i.e. branches explored before the bug was found), the average branch depth (i.e. number of decision points in each branch), and the average time taken to explore each branch (i.e. the total test time divided by the backtrack count)\footnote{
It is not clear how meaningful the average time per branch is, considering the Simics and kernel startup time are included in the test time, and Landslide is not exploring then.}.

\newcommand\nobug[1]{N/A {\em \small (#1)}}
\begin{figure*}[t!]
	\begin{tabular}{|l|l||c|c|c|c|}
		\hline
		\multirow{2}{*}{\bf Kernel and test case} & \multirow{2}{*}{\bf Property of tree} & \multicolumn{4}{|c|}{\bf Custom decision points} \\
		\cline{3-6}
		& & \bf default & \bf lock & \bf unlock & \bf both \\
		\hline\hline
		\multirow{4}{*}{POBBLES vanish/vanish (a)} & Decision points & \nobug{56} & 1296 & & \\
		& Backtracks   & \nobug{16} & 376 & & \\
		& Branch depth & \nobug{5} & 19 & & \\
		& Time/branch  & & & & \\
		\hline
		\multirow{4}{*}{POBBLES vanish/vanish (b)} & Decision points & & & & \\
		& Backtracks   & & & & \\
		& Branch depth & & & & \\
		& Time/branch  & & & & \\
		\hline
		\multirow{4}{*}{POBBLES wait/wait} & Decision points & 23 & 102 & 74 & 378 \\
		& Backtracks   & 4 & 17 & 12 & 56 \\
		& Branch depth & 6 & 10 & 9 & 14 \\
		& Time/branch  & & & & \\
		\hline
		\multirow{4}{*}{POBBLES thread\_fork/vanish} & Decision points & 24 & 394 & 273 & 2269 \\
		& Backtracks   & 5 & 70 & 56 & 410 \\
		& Branch depth & 5 & 16 & 14 & 23 \\
		& Time/branch  & & & & \\
		\hline
		\multirow{4}{*}{LudicrOS vanish/vanish} & Decision points & & & & \\
		& Backtracks   & & & & \\
		& Branch depth & & & & \\
		& Time/branch  & & & & \\
		\hline
		\multirow{4}{*}{LudicrOS yield/vanish} & Decision points & & & & \\
		& Backtracks   & & & & \\
		& Branch depth & & & & \\
		& Time/branch  & & & & \\
		\hline
	\end{tabular}
	\caption{Information about the decision trees explored when finding bugs. As in the previous table, each test case was run with the four different sets of decision points. ``\nobug{X}'' means the tree was completely explored because Landslide did not find a bug in that configuration.}
	\label{fig:trees}
\end{figure*}

\subsection{Summary of Bugs Found}

% vim: ft=tex
