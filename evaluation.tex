%%%%%%%%%%%%%%%%%%%%%%%%%%%%%%%%%%%%%%%%%%%%%%%%%%%%%%%%%%%%%%%%%%%%%%%%%%%%%%%%
\section{Evaluation}
%%%%%%%%%%%%%%%%%%%%%%%%%%%%%%%%%%%%%%%%%%%%%%%%%%%%%%%%%%%%%%%%%%%%%%%%%%%%%%%%
% 410 - talk about current state/methods (for students and for TAS)
% test suite presented
% process of instrumenting

\subsection{User Experience}

% TODO

%%%%%%%%%%%%%%%%%%%%%%%%%%%%%%%%%%%%%%%%%%%%%%%%%%%%%%%%%%%%%%%%%%%%%%%%%%%%%%%%
\subsection{Bug Case Studies}
\label{sec:eval-casestudy}

\subsubsection{Bug Descriptions}

% TODO: intro text goes here

\subsubsection{Numbers}
\label{sec:eval-numbers}

\newcommand\bugnum[2]{\textcolor{BrickRed}{{\bf #1} {\scriptsize \em (#2)}}}
\newcommand\nobugnum[2]{\textcolor{Blue}{{\em #1} {\scriptsize \em (#2)}}}

\begin{figure*}[t!]
	\begin{center}
	\small
	\begin{tabular}{|l||c|c|c|c||c|}
		\hline
		\multirow{2}{*}{\bf Kernel and test case} & \multicolumn{4}{c||}{\bf Decision points used} & \multirow{2}{*}{\bf Stress test} \\
		\cline{2-5}
		& \bf default & \bf lock & \bf unlock & \bf both & \\
		\hline\hline
		POBBLES vanish/vanish (a) & \nobugnum{31.8}{0.6} & \bugnum{57.1}{1.7} & $\infty$ & $\infty$ & \\
		\hline
		POBBLES vanish/vanish (b) & \nobugnum{32.0}{0.4} & \bugnum{51.5}{2.1} & \nobugnum{8057.9}{336.9} & $\infty$ & \\
		\hline
		POBBLES wait/wait & \bugnum{23.3}{0.7} & \bugnum{27.9}{0.8} & \bugnum{27.9}{2.1} & \bugnum{41.6}{1.4} & \\
		\hline
		POBBLES thread\_fork/vanish & \bugnum{22.0}{0.6} & \bugnum{37.4}{1.1} & \bugnum{27.6}{0.5} & \bugnum{72.0}{2.6} & \\
		\hline
		LudicrOS vanish/vanish & \nobugnum{13.2}{0.2} & \bugnum{13.7}{0.7} & \bugnum{34.6}{1.1} & \bugnum{17.1}{0.3} & \\
		\hline
		LudicrOS yield/vanish & \nobugnum{12.3}{0.3} & \bugnum{11.4}{0.4} & \nobugnum{27.4}{0.8} & \bugnum{11.7}{0.4} & \\
		\hline
		\multicolumn{6}{c}{Key: \bugnum{seconds}{stddev} indicates bug found; \nobugnum{seconds}{stddev} indicates whole tree explored with no bug.} \\
	\end{tabular}
	\end{center}
	\caption{Comparison of time taken (in seconds) to find bugs using Landslide and using conventional stress testing.
	Landslide's times are given for several different sets of decision points: the default set, consisting only of voluntary reschedules (Section~\ref{sec:components-arbiter}); and using custom decision points in addition to the default set: calls to \texttt{mutex\_lock}, calls to \texttt{mutex\_unlock}, and both.
	All numbers represent the average from 5 trials, with the standard deviations given in parentheses. ``$\infty$'' indicates the bug was not found with stress testing (after 1 hour), or that Landslide's search did not finish (after 8 hours). % TODO: after how long? what timeout?
	}
	\label{fig:numbers}
\end{figure*}

% TODO: have a graph

Table~\ref{fig:numbers} shows the time it takes to find each of these bugs using Landslide, configured with several different sets of decision points, and using conventional stress testing. The experimental set-up is as follows:

\begin{itemize}
	\item All Landslide trial times include the Simics start-up and kernel boot-up time (time between issuing the command and the test case beginning to run), roughly 15 seconds for POBBLES and 10 seconds for LudicrOS.
	\item All Landslide trials were run on the Gates-Hillman cluster machines (2.6 GHz Xeon).
	% TODO: say specs of crash machine
	\item All Landslide trials were run with ``backwards exploration'' enabled (Section~\ref{sec:using-search}).
	\item All trials were also run with Landslide configured to pay attention to only the relevant system calls (using \texttt{within\_function}; Section~\ref{sec:using-decision}).

	We believe it is reasonable to test for these bugs in this way - using the minimal set of system calls to be paid attention to as necessary to find the bug - because it follows the recommended workflow of using Landslide, which is to start with what the user judges to be the ``smallest relevant set'' of decision points. The configuration using \texttt{within\_function} was as follows.
	{\small
	\begin{itemize}
		\item POBBLES vanish/vanish(a): \texttt{within\_function vanish}
		\item POBBLES vanish/vanish(b): \texttt{within\_function vanish}
		\item POBBLES wait/wait: \texttt{within\_function wait}
		\item POBBLES thread\_fork/vanish: \texttt{within\_function thread\_fork} and \texttt{within\_function vanish}
		\item LudicrOS vanish/vanish: \texttt{within\_function vanish}
		\item LudicrOS yield/vanish: \texttt{within\_function yield} and \texttt{within\_function vanish}
	\end{itemize}
	}
\end{itemize}

%%%%

\newcommand\bugtree[1]{\textcolor{BrickRed}{\bf #1}}
\newcommand\nobugtree[1]{\textcolor{Blue}{\em #1}}
\begin{figure*}[t!]
	\begin{center}
	\small
	\begin{tabular}{|l|l||c|c|c|c|}
		\hline
		\multirow{2}{*}{\bf Kernel and test case} & \multirow{2}{*}{\bf Property of tree} & \multicolumn{4}{|c|}{\bf Decision points used} \\
		\cline{3-6}
		& & \bf default & \bf lock & \bf unlock & \bf both \\
		\hline\hline
		\multirow{4}{*}{POBBLES vanish/vanish (a)} & Decision points & \nobugtree{56} & \bugtree{1296} & N/A & N/A \\
		& Total backtracks   & \nobugtree{16} & \bugtree{376} & N/A & N/A \\
		& Average branch depth & \nobugtree{5} & \bugtree{19} & N/A & N/A \\
		\hline
		\multirow{4}{*}{POBBLES vanish/vanish (b)} & Decision points & \nobugtree{56} & \bugtree{1295} & \nobugtree{382071} & N/A \\
		& Total backtracks   & \nobugtree{16} & \bugtree{376} & \nobugtree{112706} & N/A \\
		& Average branch depth & \nobugtree{5} & \bugtree{17} & \nobugtree{16} & N/A \\
		\hline
		\multirow{4}{*}{POBBLES wait/wait} & Decision points & \bugtree{23} & \bugtree{102} & \bugtree{74} & \bugtree{378} \\
		& Total backtracks   & \bugtree{4} & \bugtree{17} & \bugtree{12} & \bugtree{56} \\
		& Average branch depth & \bugtree{6} & \bugtree{10} & \bugtree{9} & \bugtree{14} \\
		\hline
		\multirow{4}{*}{POBBLES thread\_fork/vanish} & Decision points & \bugtree{24} & \bugtree{394} & \bugtree{273} & \bugtree{2269} \\
		& Total backtracks   & \bugtree{5} & \bugtree{70} & \bugtree{56} & \bugtree{410} \\
		& Average branch depth & \bugtree{5} & \bugtree{16} & \bugtree{14} & \bugtree{23} \\
		\hline
		\multirow{4}{*}{LudicrOS vanish/vanish} & Decision points & \nobugtree{10} & \bugtree{16} & \bugtree{141} & \bugtree{42} \\
		& Total backtracks   & \nobugtree{2} & \bugtree{3} & \bugtree{48} & \bugtree{10} \\
		& Average branch depth & \nobugtree{2} & \bugtree{7} & \bugtree{9} & \bugtree{14} \\
		\hline
		\multirow{4}{*}{LudicrOS yield/vanish} & Decision points & \nobugtree{8} & \bugtree{5} & \nobugtree{149} & \bugtree{7} \\
		& Total backtracks   & \nobugtree{1} & \bugtree{0} & \nobugtree{43} & \bugtree{0} \\
		& Average branch depth & \nobugtree{2} & \bugtree{0} & \nobugtree{9} & \bugtree{0} \\
		\hline
	\end{tabular}
	\end{center}
	\caption{Information about the decision trees explored when finding bugs. As in the previous table, each test case was run with the four different sets of decision points. ``\nobugtree{X}'' means the tree was completely explored because Landslide did not find a bug in that configuration. ``\bugtree{X}'' reflects the portion of the tree that was explored before a bug was found.}
	\label{fig:trees}
\end{figure*}

Table~\ref{fig:trees} presents more detailed information about the decision trees that Landslide explored when finding these bugs.
For each set of decision points on each bug, we give the total number of decision points in the tree, the total number of backtracks (i.e. branches explored before the bug was found), and the average branch depth (i.e. number of decision points in each branch).

%%%%%%%%%%%%%%%%%%%%%%%%%%%%%%%%%%%%%%%%%%%%%%%%%%%%%%%%%%%%%%%%%%%%%%%%%%%%%%%%
\subsection{Summary of Bugs Found}

% TODO

%%%%%%%%%%%%%%%%%%%%%%%%%%%%%%%%%%%%%%%%%%%%%%%%%%%%%%%%%%%%%%%%%%%%%%%%%%%%%%%%
\subsection{Discussion}

\subsubsection{Invariants}

While evaluating Landslide on these bugs, we determined two invariants that should hold for multiple explorations on the same test case.

\begin{enumerate}
	\item {\bf Ordering invariant.} For a given set of decision points, exploring the tree in multiple different orders (``forwards''/``backwards'') should produce the same result in terms of whether a bug was found or not. The bugs found may be different, and the number of branches explored may be different, but it should never be that one ordering finds a bug while another ordering of the same tree finds no bug.
	\item {\bf Superset invariant.} For a given set of decision points, if an exploration of the resulting tree finds a bug, using a superset of that set of decision points should also find a bug. This is because the first tree will be a sub-tree of the second, as shown by never preempting at a decision point that appears in the second set but not the first.
\end{enumerate}

In short, even though Landslide may give false negatives from using imperfect sets of decision points, the exploration itself must be sound (i.e. not missing any bugs that exist in the resulting tree).\footnote{
When running LudicrOS yield/vanish with decision points on \texttt{mutex\_unlock} but not on \texttt{mutex\_lock}, we found that the ordering invariant failed - ``backwards'' exploration found no bug, while ``forwards'' exploration did. We attribute this to a bug in Landslide itself, and present the results for the backwards exploration as usual, in which Landslide found no bug.}

\subsubsection{Recommended Testing Strategies}
\label{sec:discussion-strategies}

We make several observations about the experimental results from Section~\ref{sec:eval-numbers}.

\begin{enumerate}
	\item {\bf Fewer is faster.} While it is theoretically possible that a tree built of finer-grained interleavings might encounter a bug after fewer overall backtracks, we found that this did not happen in practice.\footnote{
		The one suspicious case is the LudicrOS vanish/vanish bug - exploring the ``both'' tree found the bug faster than exploring the ``unlock'' tree, despite the latter's decision points being a subset of the former's. However, we see that exploring the ``lock'' tree found the bug faster than either, so overall the ``both'' tree did not outperform the fastest of the smaller trees.}
		In general, for two sets of decision points that both find the same bug, the one that results in shorter {\em branches} will result in fewer {\em backtracks} needed to expose the bug, and hence less overall time.
	\item {\bf Different decision points are differently likely to expose different bugs.} We see that even though the ``lock'' and ``unlock'' trees tended to be about the same size, they were each sometimes better than the other at finding bugs. The ``lock'' tree did better on the vanish/vanish bugs and the yield/vanish bug, while the ``unlock'' tree did better on the wait/wait bug and the thread\_fork/vanish bug.\footnote{
	The latter two bugs needed no more than the default set to uncover at all, but we claim this still demonstrates the principle in general.}
	\item {\bf Finding a bug is fast, if it exists.} As especially exhibited in the POBBLES vanish/vanish (b) case, if two sets of decision points generate trees of approximately equal size, but one tree contains a bug and the other doesn't, then searching the bugful tree will likely terminate much more quickly.
	Of course, it is always possible that a bug may only exist in the very last branch of a tree, but we found that in general bugs show up early during exploration.
\end{enumerate}

% TODO: Investigate the tree size that bugs did exist in. Investigate how many buggy branches there were.

In light of these, we recommend several principles to govern an overall strategy to automatically iterate through different sets of decision points in search of a bug.\footnote{
We now say ``Landslide'' here to refer to a hypothetical test framework to embody these strategies, although of course it would not have to be named that.}

\begin{enumerate}
	\item {\bf Iterate exploring, starting with smaller decision sets.} To properly test for a bug in a particular test case, Landslide should try as many different decision sets as possible.
	The first exploration should always be just the default decision set, because that tends to complete quickly, and can help identify new decision points (Section~\ref{sec:future-analysis}).
	When Landslide has multiple decision sets as candidates for the next exploration, it should prefer to explore ones that result in shorter average branch depth.
	In this way, Landslide will tend to find bugs with the minimal decision set needed to expose them, in accordance with observation 1 above.
	\item {\bf Run multiple explorations at once.}
	As per observation 2, if Landslide has two decision sets that result in roughly equal average branch depth, it cannot know in advance whether either one will find a bug and/or finish significantly faster than the other. As such, it should try to run both explorations in parallel, wait for either one to finish, and continue iterating as appropriate even if the other one has not finished.
	\item {\bf De-prioritise longer-lasting test configurations.}
	With finite resources for parallelisation, Landslide should attempt to load-balance whatever searches are running in parallel according to each one's likelihood of finding a bug.
	In light of observation 3, if a particular search is running abnormally long for its average branch depth (the POBBLES vanish/vanish (b) bug with the ``unlock'' tree is a prime example), Landslide could judge that it is less likely to end soon with a positive result, and prioritise searches with other decision sets.
\end{enumerate}

% vim: ft=tex
