\documentclass[12pt]{book}
%\usepackage{amsmath,amsthm,amssymb,fullpage,yfonts,graphicx,proof,subfig,wrapfig,appendix,hyperref,mdwlist,wasysym}
\usepackage{amsmath,amsthm,amssymb,fullpage,yfonts,graphicx,proof,appendix,hyperref,mdwlist,wasysym}
\usepackage{upgreek}
%\usepackage{times}
\usepackage[charter]{mathdesign}
\usepackage{hyperref}
\usepackage{draftwatermark}
\usepackage{algorithm}
\usepackage{algpseudocode}
\usepackage{multirow}
\usepackage[usenames,dvipsnames]{xcolor}
%\usepackage{epsfig}
\usepackage[bottom]{footmisc}
\usepackage{mjz-titlepage}
\usepackage{framed}
\usepackage{setspace}
\setstretch{1.05}
\usepackage{subfig}
\usepackage{changebar}

\begin{document}
%\captionsetup{width=.75\textwidth,font=small,labelfont=bf}

%initialize page style, so contents come out right (see bot) -mjz
\pagestyle{empty}

\title{\bf Landslide: \\
  Systematic Dynamic Race Detection in Kernel Space}
\author{Ben Blum}
\date{May 2012}
\Year{2012}
\trnumber{CMU-CS-12-118}

\committee{
Garth Gibson, Chair \\
David A. Eckhardt
}

\support{This research was sponsored by the U.S. Army Research Office under
grant number W911NF0910273.}

\disclaimer{The views and conclusions contained in
this document are those of the author and should not be interpreted as
representing the official policies, either expressed or implied, of
any sponsoring institution, the U.S. government or any other entity.}

% copyright notice generated automatically from Year and author.
% permission added if \permission{} given.

\keywords{concurrency, kernel debugging, race conditions, runtime verification}

\maketitle

\begin{dedication}
\begin{tabular}{p{6in}}
\centering
There is a fractal nature to all aspects of life.

At the time of writing, I am focused entirely on this one project. There is a vast amount of thought in this thesis, and many more rich worlds lie hidden in each of the open avenues for future work.

Yet it pales in comparison to the scope of a Ph.D. thesis, and even more so next to all the details of a person's entire life.
After I finish this project, it will become but a fond memory, a mere building block in a city of even grander things.

Human nature is beautiful in its ability to exist at any level of perspective, to hold in awareness a story of any size, and to zoom in and out at will to see how multiple parts fit together.
Tearing ourselves up over minutiae is part of the fun, and contemplating overarching directions beyond our ability to change is also part of the fun.

Our true home, though, lies somewhere in between: a place of potential, and of peace. May we always remember to return to it.
\end{tabular}
\end{dedication}

\pagestyle{plain} % for toc, was empty


\newcommand\true{\;\textit{true}}
\newcommand\false{\;\textit{false}}

\newcommand\alpher\alpha
\newcommand\beter\beta
\newcommand\gammer\gamma
\newcommand\delter\delta
\newcommand\zeter\zeta
\newcommand\Sigmer\Sigma

\newcommand\NN{\mathbb{N}}
\newcommand\QQ{\mathbb{Q}}
\newcommand\RR{\mathbb{R}}
\newcommand\ZZ{\mathbb{Z}}

\setcounter{changebargrey}{35}
\newcommand\revision[1]{\begin{samepage}\cbstart#1\cbend\end{samepage}}
%\newcommand\revision[1]{#1}

\begin{abstract}
Systematic exploration is an approach to finding race conditions by deterministically executing every possible interleaving of thread transitions and identifying which ones expose bugs.
Current systematic exploration techniques are suitable for testing user-space programs, but are inadequate for testing kernels, where the testing framework's control over concurrency is more complicated.

We present Landslide, a systematic exploration tool for finding races in kernels.
Landslide targets Pebbles, the kernel specification that students implement in the undergraduate Operating Systems course at Carnegie Mellon University (15-410).
We discuss the techniques Landslide uses to address the general challenges of kernel-level concurrency, and we evaluate its effectiveness and usability as a debugging aid.
We show that our techniques make systematic testing in kernel-space feasible, and that Landslide is a useful tool for doing so in the context of 15-410.
\end{abstract}

\tableofcontents
%%%%%%%%%%%%%%%%%%%%%%%%%%%%%%%%%%%%%%%%%%%%%%%%%%%%%%%%%%%%%%%%%%%%%%%%%%%%%%%%
\chapter*{Acknowledgements}
%%%%%%%%%%%%%%%%%%%%%%%%%%%%%%%%%%%%%%%%%%%%%%%%%%%%%%%%%%%%%%%%%%%%%%%%%%%%%%%%

This has been an immensely fun and fulfilling project, and it wouldn't have been possible without the contributions, guidance, and support of many people.

\subsubsection{Big Thanks}

First of all, I owe the lion's share of my gratitude equally to Garth Gibson, Jiri Simsa, and David Eckhardt.

Garth's advisorship over the past year has been excellent, from showing me how to keep my research both useful and personally satisfying, to teaching me how to share ideas with fellow researchers, to helping me think through far-off future work directions. Not working directly with his main research groups was sometimes unsettling, but validation was not in short supply: for example, it was definitely a high point when Garth emailed me unprompted asking for my input on the future of my research field.

Jiri's mentorship was invaluable, both in explanations of complicated algorithms and in general advice on good researcher habits. I would have had a much harder time understanding Dynamic Parial Order Reduction without him, and I suspect my posters and talks would have been much less put-together as well.

Working with David to share my project with the students of 15-410 really made the thesis come together. I knew I would have a rough time making Landslide available without compromising the class's integrity policies, and David helped me make it work. His advice was also crucial when designing the lecture I gave to the class to explain Landslide and recruit students for the user study; I learned more about lecture design during those two weeks than I ever thought there was to know.

\subsubsection{Landslide's Users}

I'm also highly grateful for all the patient work put in by the kernel programmers who used Landslide.

Nathaniel Filardo, as the first person to use Landslide besides myself, ran into several compatibility issues (and outright bugs) in Landslide when instrumenting his kernel. For all Landslide gave him a hard time, though, he stayed with me, was patient while I debugged, and even debugged some on his own when I was too busy myself. It was when he got Landslide working with his kernel that I finally found confidence that it'd be able to work on arbitrary student kernels.

Thanks also to the other members (and former members) of 15-410 staff who used Landslide on their kernels to help me prepare for working with the students: Josiah Boning, Alex Crichton, and Michael Sullivan.

A special form of thanks to the students I worked with, who not only had to navigate Landslide's interfaces but also had to learn the underlying concepts as well, all while under time pressure to finish their kernel projects at the same time. To Nadim Taha, Margaret Schervish, Timothy Passaro, Joon-Sup Han, Pranjal Jumde, and others who did not ask to be mentioned by name: I could not have evaluated Landslide without you.

Another piece of thanks goes to Nadim Taha for his above-and-beyond investigation of the decision trees Landslide explored using his kernel, which resulted in the insights presented in Section~\ref{sec:future-nadim}.

\subsubsection{Collaboration}

Thanks to Wind River for Simics\textsuperscript{\texttrademark}, the simulation software for which Landslide is built. While developing code to fit into larger infrastructure always has its hairy points, Simics's interfaces were generally pleasant, and the expressive power of operating in its simulation environment was invaluable.

Thanks to the members and companies of the PDL Consortium (including Actifo,
APC, EMC, Emulex, Facebook, Fusion-io, Google, Hewlett-Packard, Hitachi, Huawei
Technologies, Intel, Microsoft, NEC, NetApp, Oracle, Panasas, Riverbed,
Samsung, Seagate, STEC, Symantec, VMware, and Western Digital) for their
interest, insights, feedback, and support.
It has been a pleasure to work in this research group.

\subsubsection{Support}

More gratitude goes to the people who helped me revise my thesis document and defence talk. While Garth and David provided the most detailed criticism, many of my friends helped me polish my work: Wolfgang Richter, Carlo Angiuli, Joshua Wise, Ryan Pearl, and Michael Sullivan critiqued this document, and Jiri Simsa, Matthew Maurer, Greg Hanneman, and Carlo Angiuli critiqued my defence presentation.

Finally, huge thanks to Deborah Cavlovich and Catherine Copetas, whose organisational efforts carried me through the Master's program. These are the people who make the gears turn under the hood.

\vspace{0.5in}

\noindent I hope you enjoy reading some or all of this.


%%%%%%%%%%%%%%%%%%%%%%%%%%%%%%%%%%%%%%%%%%%%%%%%%%%%%%%%%%%%%%%%%%%%%%%%%%%%%%%%
\chapter{Introduction}
%%%%%%%%%%%%%%%%%%%%%%%%%%%%%%%%%%%%%%%%%%%%%%%%%%%%%%%%%%%%%%%%%%%%%%%%%%%%%%%%

Race conditions are notoriously difficult to debug.
Because of their nondeterministic nature, they frequently do not manifest at all during testing, and when they do manifest, it is difficult to reproduce them reliably enough to collect enough information to help debugging.

\revision{
Many techniques exist for dynamic testing of concurrent systems for race conditions.
Systematic exploration\cite{chess}, the strategy we focus on in this work, involves making educated guesses at what points during execution a preemption would be most likely to expose a bug, enumerating the different possibilities for interleaving threads around these points, and forcing the system to execute all such interleavings to check if any of them results in incorrect behaviour.
Systematic exploration provides a better alternative to conventional long-running stress tests, because it is less likely to overlook buggy execution patterns, and it enables a testing framework to report more thorough debugging information. Compared to other dynamic analyses, such as data race detection\cite{datacollider}, systematic exploration is able to find a wider range of types of concurrency errors because of its ability to manipulate the execution of the system under test.

In kernel space, race condition debugging becomes even more difficult. Many aspects of the concurrency implementation itself are part of the system being tested. While user-space testing frameworks rely on the underlying kernel to provide common concurrency abstractions which the framework can use to drive the test, systematic exploration in the kernel itself necessarily exists underneath the level of such abstractions.
Furthermore, a tool that supports testing multiple kernels must support many different possible scheduler designs.

\begin{quote}
\bf
Systematic exploration in kernel space is possible by understanding a kernel's concurrency through instrumentation of its internal abstractions, and can be made efficient by relying on the user to control the scope of the test.
\end{quote}
}

We present Landslide,\footnote{
Landslide {\em (n)}: a phenomenon which demonstrates that Pebbles are not as stable as you might think.}
a tool for applying systematic race detection techniques in to kernel-level code.
It is geared towards kernels that implement the Pebbles specification, the main project in Operating System Design and Implementation (15-410) at CMU, and implemented as a module for Wind River Simics\textsuperscript{\texttrademark}, the x86 simulator that students use to run their Pebbles kernels.
During execution of a kernel, Landslide records important actions performed by the kernel, attempts to decide at which points in the kernel's execution a preemption will be most likely to expose a bug, and then exercises all possible interleavings of kernel threads around such points.
When Landslide finds a bug (predicted by a set of common checks and heuristics), it stops execution and prints information about the sequence of interleavings that exposed the bug.

With Landslide, we see testing a kernel as a process of manipulating test parameters in two ways: first, in the choice of test case (the user-space program that exercises a specific set of system calls), and second, in the configuration of Landslide in regard to which parts of the kernel are ``interesting'' in the behaviour of the test case and which are irrelevant.
Searching for and understanding race conditions exposed by a given test becomes a joint effort between the programmer and Landslide, combining the programmer's specific knowledge about the design of the kernel and Landslide's ability to explore many interleavings efficiently.

\revision{
In this work, we study the challenges inherent in applying systematic testing techniques in kernel-space, in contrast with user-space applications (Chapter~\ref{sec:challenges}), present our techniques (some sound, and some heuristic) for addressing them (Chapter~\ref{sec:design}), and document the process of using Landslide (Chapter~\ref{sec:using}).
We evaluate Landslide both by studying its potential to be a helpful debugging aid in 15-410 and by studying certain bugs in detail to determine effective usage strategies (Chapter~\ref{sec:evaluation}).
Finally, we discuss possible avenues for future work in education, use of additional testing techniques, application to general purpose kernels, long-running testing approaches, and theoretical study of systematic exploration (Chapter~\ref{sec:future}).
}

%%%%%%%%%%%%%%%%%%%%%%%%%%%%%%%%%%%%%%%%%%%%%%%%%%%%%%%%%%%%%%%%%%%%%%%%%%%%%%%%
\section{Related Work}
%%%%%%%%%%%%%%%%%%%%%%%%%%%%%%%%%%%%%%%%%%%%%%%%%%%%%%%%%%%%%%%%%%%%%%%%%%%%%%%%

Systematic exploration is a relatively young approach to concurrency verification.
Recent work has focused on theoretical aspects of the technique, on distributed systems, and in userspace in general.
Other work has focused on kernel verification apart from systematic exploration, using techniques such as data race detection and formal model verification.
Still other efforts are for testing and verification techniques that are orthogonal to systematic exploration and kernel-level testing entirely.
Here we discuss related work in all three categories.

Landslide distinguishes itself from related systematic testing tools in its field as more of an exploratory work, presenting techniques to make systematic testing compatible with kernel-level code; less so as a development on the core testing approach itself.

\subsection{Systematic Exploration}

Dynamic Partial Order Reduction (DPOR) \cite{dpor,sdpor,dbug-retreat} is an algorithm for reducing the state space of ``all possible thread interleavings'' by identifying independent thread transitions and pruning the redundant state spaces that result from interleaving those transitions. Landslide makes generous use of the DPOR algorithm.

dBug\cite{dbug-ssv} is a systematic testing tool which makes use of DPOR. It focuses on multithreaded userspace applications, identifying when to preempt by interposing on \texttt{libc} library calls. It uses a message-passing model for inter-thread communication and for establishing a happens-before relationship between transitions. The dBug project heavily inspired the development of Landslide.

CHESS\cite{chess} is another userspace systematic testing tool. They explore the insight that many races require very few forced preemptions to uncover, and develop a search strategy which prioritises thread interleavings with fewer preemptions. Landslide does not (yet) make use of this strategy, though we did find a related insight which we discuss in Section~\ref{sec:future-backwards}.

There are also several tools, such as MaceMC\cite{macemc} and MODIST\cite{modist}, which provide systematic testing frameworks for networked and distributed applications.

DeMeter\cite{demeter} is a more recent tool for systematic exploration that introduces Dynamic Interface Reduction as a strategy for constraining the size of the state space, which means targetting the testing technique on only one component of the program at a time. Landslide makes use of the same general concept in its user interface (Section~\ref{sec:using}) - it provides the user with options to constrain decision point identification to certain subsets of the kernel.

\subsection{Kernel-level Verification}

\cite{racepro}
\cite{sel4}
\cite{datacollider}
\cite{carburizer}

\subsection{Orthogonal Testing Techniques}

Research in dynamic verification has also seen other techniques apart from systematic exploration.

Symbolic execution\cite{symbolic,symbolic-disks} is a technique for testing programs by abstractly interpretating symbols and operations within a program, rather than directly executing its compiled code, which grants the ability to analyse conditional statements and cause-effect relationships among certain code paths.
Tools such as KLEE\cite{klee} and projects such as Automated Exploit Generation\cite{aeg} demonstrate the effectiveness of symbolic execution in dynamic verification.
We discuss the possibility of integrating symbolic execution into a testing framework such as Landslide in Section~\ref{sec:future-new}.

Deterministic multithreading is a different technique for dealing with concurrency bugs which focuses on avoiding them when running production code rather than trying to expose them during testing. Deterministic multithreading tools, such as Kendo\cite{kendo}, PEREGRINE\cite{peregrine}, and DThreads\cite{dthreads}, analyse the execution of a concurrent system and compute particular scheduling patterns which will not produce buggy behaviours, and force the system to follow those schedules while maintaining good parallel performance.
Deterministic multithreading serves a different purpose than systematic exploration; it aims to ensure that code already running in the real world does not encounter concurrency bugs even though they might exist, while our goals are instead to uncover such bugs and help fix them beforehand.

% vim: ft=tex
 % COMPLETE!
%%%%%%%%%%%%%%%%%%%%%%%%%%%%%%%%%%%%%%%%%%%%%%%%%%%%%%%%%%%%%%%%%%%%%%%%%%%%%%%%
\chapter{Challenges of Kernel Space}
%%%%%%%%%%%%%%%%%%%%%%%%%%%%%%%%%%%%%%%%%%%%%%%%%%%%%%%%%%%%%%%%%%%%%%%%%%%%%%%%
\label{sec:challenges}

\section{Causes of Concurrency}

In user space, a systematic exploration tool may cooperate with the underlying kernel to help control the concurrent behaviour of the system\cite{dbug-ssv}. Simple system call invocations can cause a particular thread to run at a particular time, or to block while another thread runs first.

In kernel space, however, the scheduler is part of the system being tested, and we can no longer always interrupt the execution of a test case to ask the scheduler to instantly start running a different thread.
\begin{itemize}
	\item {\bf Non-preemptibility.} Certain regions of code may be non-preemptible, so a testing tool must know when it is legal to preempt a kernel thread.
	\item {\bf The context switcher.} A context switch between threads is no longer ``instantaneous'': many instructions must be run between when we decide to preempt and when the next thread begins running, and the tool must be aware of this ``intermediate state''. A tool must be aware of the context switch process both to avoid flooding the kernel with interrupt frames that would overflow the stack, and to be able to ignore shared memory conflicts inherent to all thread transitions (Section~\ref{sec:por-independence}).
	\item {\bf Run-queue tracking.} The kernel's scheduler and the tool must cooperate in some way so that the tool can both know what threads are runnable at every point during execution and cause any given runnable thread to begin running in place of the current one.
\end{itemize}

\begin{figure*}[t]
	\begin{center}
	\includegraphics[width=0.65\textwidth]{vip1066720.jpg}
	\end{center}
	\caption{{\em !`Cuidado! !`Las llamas son muy peligrosas!}\cite{de0u}}
\end{figure*}

\section{Ad-hoc Thread Communication}

Inter-thread communication can be much more ad-hoc in kernel-space than in user-space\hspace{0in}\cite{datacollider}, and a systematic testing framework needs to be aware of all the kernel's synchronisation idioms.

\begin{itemize}
	\item {\bf Avoiding reliance on message-passing.} Some systems for user-space systematic testing require that threads communicate only by message-passing, to better track the concurrency relationships between thread transitions\cite{dbug-ssv}. With few exceptions\cite{barrelfish}, kernels do not rely on message-passing as a primary communication mechanism, and to be compatible with the kernels of today, a testing framework must allow for less idealised state-sharing (Section~\ref{sec:components-memory}).
	\item {\bf Recognising blocked threads.} It is difficult to detect when threads are ready to do useful work. Even if notionally runnable, they may be waiting in a yield loop (Sections~\ref{sec:using-annotations} and \ref{sec:future-linux}), which can be difficult to discern without an understanding of whatever shared resource the thread is waiting for. A tool that did not recognise this might decide to attempt to run that thread over and over, and get stuck because the other threads would never be allowed to make progress.
	\item {\bf Thread lifecycle tracking.} It is also difficult to demarcate threads' lifecycles: when a new thread gets created, when is it available to run? When a thread is exiting, at what point does it stop running code (Section~\ref{sec:using-annotations})? In user-space, these boundaries are defined by the system call instructions, but in kernel-space, the boundary is fuzzier.
	\item {\bf Use of virtual memory.} The kernel's complete control over the machine's virtual memory system means a system for tracking memory accesses must be aware of changing virtual memory mappings (Section~\ref{sec:using-requirements}). A tool not aware of virtual memory might falsely identify whether or not two transitions' memory accesses conflict.
\end{itemize}

\section{Kernel Design Independence}
\label{sec:challenges-design}

One challenge particular to the context of 15-410 is that the kernels Landslide must be able to test may all use slightly different implementations to achieve the same goals. For Landslide to be generally applicable to (almost) any kernel that students may submit, it must make some abstract assumptions about the kernel design which are compatible with many different implementations.

% In certain avenues of future work (Section~\ref{sec:future-linux}), remaining agnostic of most implementation details becomes less of an issue. For example, if we target Landslide to Linux, it is unlikely that core scheduler or interrupt-handling behaviour will change for as long as Landslide remains useful for finding bugs. In 15-410, however, repeated re-targetting is a necessity.

We built Landslide to be compatible with multiple designs for the following major scheduling behaviours:

\begin{itemize}
	\item {\bf Runqueues}: Does the kernel store the currently-running thread on the runqueue, or is it ``checked out'' while running and stored separately?
	\item {\bf Mutexes}: When a thread blocks on a mutex, is it left on the runqueue in a yield loop, or is it explicitly descheduled (e.g., moved from the runqueue to another queue)? In yield-looping mutexes, when do blocked threads become ``unblocked'', notionally? (This might happen before the blocked thread runs next.)
	\item {\bf Idling}: Does the kernel have an explicit idle thread, or is there an idle loop that runs on the stack of whatever thread was last running? Do explicit idle threads run their idle loop in userspace or in kernelspace?
	\item {\bf Thread Creation}: When a thread is newly forked, is it placed onto the runqueue for later, or is it context-switched to immediately? Do just-forked threads begin life through the usual context-switch-return path, or is there a special path for that?
	\item {\bf Test Lifecycle}: Apart from running on a particular input that the user wishes to test, the kernel may need to perform extra work irrelevant to the test, such as initialisation (e.g., the bootup process) and housekeeping (e.g., cleaning up dead processes). Which parts of this work should be included in or excluded from the test?
\end{itemize}
 % COMPLETE!
%%%%%%%%%%%%%%%%%%%%%%%%%%%%%%%%%%%%%%%%%%%%%%%%%%%%%%%%%%%%%%%%%%%%%%%%%%%%%%%%
\chapter{Terminology}
%%%%%%%%%%%%%%%%%%%%%%%%%%%%%%%%%%%%%%%%%%%%%%%%%%%%%%%%%%%%%%%%%%%%%%%%%%%%%%%%
\label{sec:terminology}

In this chapter we define certain terms that have specific meanings in the context of Landslide.

\section{Basic Terms}

\begin{enumerate}
	\item {\bf Guest kernel}:
		The kernel which is being tested for concurrency bugs by Landslide.
	\item {\bf Test case}:
		Abstractly, the set of inputs under which the guest kernel is tested. Practically, a user-space program which runs on top of the guest kernel to execute a particular set of system calls, while Landslide performs systematic exploration.
\suspend{enumerate}

\section{Scheduling Terms}

\resume{enumerate}
	\item {\bf Thread}:
		One participating agent in a concurrent program, which executes code sequentially, with the potential to be interleaved with the execution of other threads in the system.
		Each thread has a unique numeric identifier (TID).
	\item {\bf Involuntary preemption}:
		A context switch from one thread to another caused by a nondeterministic event, such as a timer tick or device interrupt. The systematic testing framework must control all sources of nondeterminism. In this work, we focus only on timer-driven thread switches.
	\item {\bf Voluntary reschedule}:
		A context switch from one thread to another which the guest kernel performed automatically, not triggered by nondeterministic events and/or Landslide but rather as part of its normal execution.\footnote{Voluntary reschedules may or may not be necessary for the kernel to behave correctly at all (for example: switching away from an exiting thread or from a thread blocked on \texttt{wait} is necessary, but switching to a newly-forked thread is not). Involuntary preemptions are never necessary for correct kernel behaviour.}
		See Section~\ref{sec:components-arbiter}.
\suspend{enumerate}

\section{Systematic Exploration Terms}

\begin{figure}[h]
	\centering
	\includegraphics[width=0.75\textwidth]{threadfork.png}
	% TODO: make some other subfigs
	\caption{A simple decision tree, showing possible interleavings of a buggy \texttt{thread\_fork} implementation. (This describes a bug studied in Section~\ref{sec:eval-thread-fork}.)}
	\label{fig:threadfork}
\end{figure}

\resume{enumerate}
	\item {\bf Decision point}:
		A point during execution (between two consecutive instructions, precisely) which is deemed ``interesting'' in terms of the likelihood that thread switches at that point will cause concurrency bugs to arise. All thread switches occur at a decision point; some are voluntary reschedules caused by the kernel and others are involuntary preemptions caused by Landslide.
	\item {\bf Transition}:
		A sequence of instructions executed by a single thread between two decision points. At a given decision point, Landslide chooses which thread will run (and causes it to do so), and that thread executes a transition, after which point Landslide will have identified a subsequent decision point
	\item {\bf Decision set} (or {\bf Set of decision points}):
		One or more predicates on the execution state of the guest kernel which identify whether any current state should be a decision point. (For example, when we say ``the decision set including all calls to \texttt{mutex\_lock}'', the corresponding predicate is simply ``did the kernel just invoke \texttt{mutex\_lock}?''.)
	\item {\bf Decision tree} (or {\bf Execution tree}):
		The tree defined by the possibility at each decision point of causing any runnable thread to execute its own transition. Abstractly, this tree comprises all possible states reachable by any interleaving of threads.
		A decision tree is created/discovered during an exploration using a particular decision point set; the set is what the user configures, and the tree is what arises as a result.
	\item {\bf Branch}:
		One particular execution of the test case; a set of decision points with exactly one transition in between each of them, characterising a single interleaving of threads.
	\item {\bf Interleaving}:
		A less precise / more abstract term for a branch, or for a subset of transitions which make up a branch.
	\item {\bf Bug}: An execution state which Landslide identifies as incorrect, which is indicative of illegal behaviour. Each branch of the tree may or may not have a bug in it. See Section~\ref{sec:techniques-bugs}.
	\item {\bf Backtracking}:
		Performed at the end of each branch of the tree, when the test case has finished running. Landslide identifies a decision point from the current branch at which it wants to have chosen a different thread to run, reverts the state of the guest kernel to what it was at that decision point, and causes a different thread to run, hence exploring a different branch/interleaving.
	\item {\bf Decision trace}:
		A list of information about all decision points in a given branch, printed when a bug is found to help the user analyse its cause. Contains, for each decision point, the TID of the thread that was running previously, the TID that was newly chosen, the current instruction pointer, and the stack trace of the previous thread at the point it was switched away from.\footnote{Though Landslide does not provide it, a decision trace may also contain for each transition a list of shared memory accesses that conflicted with other transitions, and/or a listing of happens-before relationships between transitions.}
\end{enumerate}

% vim: ft=tex
 % almost complete
%%%%%%%%%%%%%%%%%%%%%%%%%%%%%%%%%%%%%%%%%%%%%%%%%%%%%%%%%%%%%%%%%%%%%%%%%%%%%%%%
\chapter{Design and Implementation}
%%%%%%%%%%%%%%%%%%%%%%%%%%%%%%%%%%%%%%%%%%%%%%%%%%%%%%%%%%%%%%%%%%%%%%%%%%%%%%%%
\label{sec:design}

In this chapter we describe Landslide's implementation in detail. Section~\ref{sec:model} outlines the testing model on top of which Landslide's exploration mechanisms are built. Section~\ref{sec:components} describes the individual components within Landslide. Section~\ref{sec:techniques-bugs} describes Landslide's metrics for identifying bugs. Section~\ref{sec:techniques-por} describes how Landslide achieves effective state space reduction. Section~\ref{sec:techniques-feedback} describes the debugging feedback Landslide provides to the user when bugs are encountered.

\revision{
\section{Landslide's View of the World}
\label{sec:model}

Landslide's model for applying systematic exploration in kernel-space is comprised of four key points, which we overview here.

\subsection{Simulated Execution}
\label{sec:model-simics}

Landslide is implemented as a module for Simics\cite{simics}, a full-system x86 emulator. When running the kernel, Simics calls into Landslide once every time the kernel executes an instruction or performs a memory read or write. Landslide uses this information, in conjunction with the user-provided instrumentation, to maintain its internal representations of the state of the guest kernel.

Landslide makes use of Simics ``bookmarks'', a feature which enables checkpointing and restoring the execution state of the guest kernel, to implement backtracking when the end of each branch in the decision tree is reached.

\subsection{Timer-Driven Scheduling}
\label{sec:model-timer}

In this work, we focus specifically on nondeterministic scheduling driven by timer interrupts. Landslide assumes that timer interrupts are the only source of nondeterminism for the guest kernel, so controlling when they occur theoretically allows for complete control over the concurrent behaviour of the test case. In future work (Section~\ref{sec:future-drivers}), we may also address causes of non-determinism more complex than timer-driven thread scheduling, such as interrupts and data I/O from peripheral devices.

Landslide needs to impose some requirements on the guest kernel's scheduling behaviour, in order for its control over scheduling to work:

\begin{itemize}
	\item {\bf Timer ticks control ``runnable'' threads.} With the exception of a non-preemptible ``scheduler lock'', and yield-looping mutexes (both of which must be instrumented by the user), a thread's presence on the runqueue indicates that a finite number of timer interrupts in succession will eventually cause it to run.

	Landslide treats the sleep queue no differently from the runqueue when deciding which threads are runnable. The guest kernel will treat it differently, but since sleeping for a predetermined amount of time is never an appropriate way to solve race conditions, Landslide treats sleeping threads as notionally runnable. This fits directly into its model that runnable threads are ones that can be caused to run with a finite number of timer ticks in succession.
	\item {\bf No idling when progress can be made.} The kernel must not enter its idle loop (whether in an explicit idle thread or no) when the kernel is not truly idle. Landslide uses the idle loop to detect when a test begins/ends (test lifecycle tracking) and to detect when all threads in a test are wedged (bug detection).
\end{itemize}

Section~\ref{sec:components-inflight} gives more detail about how Landslide abstracts raw timer interrupts into the ability to cause arbitrary threads to run.

\subsection{False-Negative-Oriented Bug Detection}
\label{sec:model-bugs}

% A healthy dose of snark.
Without a formal specification of the internals of the guest kernel's implementation\cite{sel4}, it is impossible to identify both soundly and completely when a behaviour that constitutes a ``bug'' arises during a test case's execution.
Landslide's bug reporting is false-negative oriented, meaning that it does not check for suspicious behaviours that might indicate underlying bugs, so it may report that it found no bugs even if some existed. If Landslide does report a bug, though, it is almost certainly correct. Section~\ref{sec:techniques-bugs} details the conditions Landslide uses to identify when something has gone wrong.

\subsection{User-Assisted State Space Reduction}
\label{sec:model-user}

In addition to Dynamic Partial Order Reduction (DPOR) for automated state space reduction, Landslide relies on the user's guidance to help mitigate the possible combinatorial explosion of thread interleaving possibilities. This manifests in two ways:

\begin{itemize}
	\item {\bf Decision point selection.} Landslide provides an interface for the user to manually limit identification of decision points to only ``relevant'' modules of the kernel, thereby only generating thread interleavings around points within those modules.
	\item {\bf Memory conflict selection.} Landslide also provides a mechanism for ignoring shared memory conflicts on certain common data structures (especially those accessed in every thread transition, such as the scheduler queues), to enable DPOR to achieve greater reduction. Intuitively, this represents sacrificing the ability to find races around those data structures to make searching other components more feasible.
\end{itemize}

We advocate relying on the user for such information because the user need only understand the basics of their kernel design (which is difficult for Landslide to guess) while not knowing in advance what bugs are being looked for (which Landslide is able to provide). We claim that this combination of the user's knowledge and Landslide's testing mechanisms leads to an effective usage dynamic, in which the user ``steers'' Landslide towards focused search spaces that are more likely to find bugs.

The interfaces by which the user provides this information are documented in Section~\ref{sec:using-customise}.
}

\section{Components of Landslide}
\label{sec:components}

In this section we briefly describe each major component of Landslide.

\begin{figure}[h]
	\centering
	\includegraphics[width=0.75\textwidth]{landslide.png}
	\caption{A simple visualisation of Landslide and its components, and how they interact with the guest kernel.}
	\label{fig:landslide}
\end{figure}

\subsection{Kernel Instrumentation}
\label{sec:components-kern}

The kernel instrumentation serves as the glue between what the guest kernel is doing and Landslide's understanding of the guest kernel's state.

\begin{itemize}
	\item {\bf User-provided instrumentation.} Some parts of the kernel may be written in any number of ways, and hence require the user's assistance for Landslide to understand. The user-provided instrumentation, described in detail in Section~\ref{sec:using-instrumenting}, informs Landslide about thread-related lifecycle and scheduling events and the anatomy of the kernel's scheduler.
	\item {\bf Automatic instrumentation.} All Pebbles kernels have some things in common (largely due to the common starter code provided for the class project), and Landslide's build system is able to automatically instrument certain parts of the kernel. This instrumentation informs Landslide about the dynamic memory allocator, common library functions such as \texttt{panic}, and certain aspects of the kernel's executable format.
\end{itemize}

\subsection{Scheduling}
\label{sec:components-sched}

The Landslide scheduler is responsible for keeping track of which threads exist in the guest kernel: which are runnable at any given time, and when they are created and destroyed.

It maintains a ``mirror image'' of the guest kernel's scheduler state in the form of three queues, a pointer to the currently-running thread, and a pointer to the previously-running thread. The queues are the {\em runqueue}, containing the runnable threads, the {\em sleep queue}, containing threads which become runnable after a certain number of timer ticks, and the {\em deschedule queue}, which might not correspond to a data structure in the guest kernel, but contains all other threads that exist on the system, which are not runnable for whatever reason.

The Landslide scheduler also tracks which important actions each thread is performing. These actions are {\em forking}, {\em vanishing}, and {\em sleeping}, which are described in the next section \revision{(note that ``vanish'' is simply Pebbles's name for ``exit'')}, and also {\em in\_timer\_handler} and {\em in\_context\_switch}, which express what type of context switch a given thread may be performing, and are useful for both thread scheduling (Section~\ref{sec:components-inflight}) and detecting voluntary reschedules (Section~\ref{sec:components-arbiter}).\footnote{
There are also other minor actions: {\em readlining}, used to track test lifecycle (Section~\ref{sec:components-test}), {\em just\_forked}, used for special-case context switch behaviour (Section~\ref{sec:components-arbiter}), and also several flags for tracking mutex lock/unlock events.}

\subsubsection{Thread Lifecycle Tracking}

\begin{algorithm}[t]
	\footnotesize
	\begin{algorithmic}
		\State {\em Per-thread state.}
		\State bool forking, sleeping, vanishing, in\_timer\_handler;
		\State int current\_tid;
		\\
		\Function{handle\_tell\_landslide\_forking}{}
			\State forking $\gets$ true;
		\EndFunction
		\\
		\Function{handle\_tell\_landslide\_sleeping}{}
			\State sleeping $\gets$ true;
		\EndFunction
		\\
		\Function{handle\_tell\_landslide\_vanishing}{}
			\State vanishing $\gets$ true;
		\EndFunction
		\\
		\Function{handle\_tell\_landslide\_thread\_switch}{int new\_tid}
			\If {!in\_timer\_handler}
				\If{sleeping}
					\State {\sc add\_to\_sleep\_queue}(current\_thread);
					\State sleeping $\gets$ false;
				\ElsIf{vanishing}
					\State {\sc destroy\_thread}(current\_tid);
					\State vanishing $\gets$ false;
				\ElsIf{forking}
					\State {\sc create\_thread}(new\_tid);
					\State forking $\gets$ false;
				\EndIf
			\EndIf
			\State {\sc update\_current\_thread}(new\_tid);
		\EndFunction
	\end{algorithmic}
	\caption{Landslide's scheduler's routines for tracking thread lifecycles. These routines are invoked each time the guest kernel calls one of the \texttt{tell\_landslide} annotations.}
	\label{alg:tell-landslide}
\end{algorithm}

The Landslide scheduler relies on the guest kernel informing it about certain important events in the thread lifecycle. The guest kernel does this by means of annotations (which are described in Section~\ref{sec:using-annotations}). Algorithm~\ref{alg:tell-landslide} shows how Landslide handles these annotations internally.\footnote{
There are also cases for the runqueue annotations, \texttt{tell\_landslide\_on\_rq} and \texttt{tell\_landslide\_off\_rq}, which we omit for brevity. One important note is that the routine for \texttt{tell\_landslide\_on\_rq} must also check the ``forking'' flag, in case the kernel's \texttt{fork} implementation does not immediately switch to the new thread but instead adds it to the runqueue for later.}

\subsubsection{Thread Scheduling}
\label{sec:components-inflight}

Finally, the Landslide scheduler is responsible for managing involuntary preemptions, causing arbitrary threads to begin running in place of the current one (as chosen at decision points and the ends of branches by the explorer, Section~\ref{sec:components-explore}).

\begin{algorithm}[t]
	\footnotesize
	\begin{algorithmic}
		\State {\em Global scheduler state.}
		\State bool schedule\_in\_flight;
		\State int target\_tid;
		\\
		\State {\em Per-thread state. (Updated elsewhere.)}
		\State bool in\_timer\_handler, in\_context\_switch;
		\State int current\_tid;
		\\
		\Function{scheduler\_update}{int pc}
			\If{schedule\_in\_flight}
				\State {\sc assert}(in\_timer\_handler || in\_context\_switch);
				\If { {\sc kernel\_exiting\_timer}(pc) || (!handling\_timer \&\& {\sc kernel\_exiting\_context\_switch}(pc))}
					\State {\em The kernel has just finished rescheduling and is about to resume normal thread execution.}
					\If {current\_tid != target\_tid}
						\State {\em The kernel switched to an undesirable thread. Keep the schedule operation ``in-flight''.}
						\State {\sc cause\_timer\_interrupt}();
					\Else
						\State {\em The in-flight schedule is ``landing''.}
						\State schedule\_in\_flight $\gets$ false;
					\EndIf
				\EndIf
			\ElsIf{{\sc need\_to\_preempt}()}
				\State target\_tid $\gets$ {\sc choose\_new\_thread}();
				\State schedule\_in\_flight $\gets$ true;
				\State {\sc cause\_timer\_interrupt}();
			\EndIf
		\EndFunction
	\end{algorithmic}
	\caption{Landslide's scheduling algorithm. This procedure for updating Landslide's state is executed once per instruction, with a corresponding value for \texttt{pc} (the program counter) each time. The predicates on \texttt{pc} are part of the kernel instrumentation (Section~\ref{sec:components-kern}).}
	\label{alg:inflight}
\end{algorithm}

Though we define timer interrupts as the only source of non-determinism in our environment, it is more useful to view the concurrent behaviour with a higher-level abstraction, in terms of the set of runnable threads and the ability to preempt the currently-running thread with any different runnable one.

Landslide's scheduling technique, called the {\em schedule in-flight}, involves successive triggering of timer interrupts to trigger context switches until the desired thread begins to run. Algorithm~\ref{alg:inflight} shows how Landslide makes this happen.

One alternative simpler method would be, when triggering a timer interrupt, to tell the guest kernel explicitly which thread should be run next. This would require the kernel programmer to write extra code in their timer handler and/or context switcher. We chose the ``in-flight'' approach instead, because it does not require additional kernel modifications.

\subsection{Memory Tracking}
\label{sec:components-memory}

Landslide maintains a mirror image of the guest kernel's dynamic allocation heap, so it can know at any point which memory ranges are allocated and which ranges used to be allocated but now are freed. This set is updated each time the guest kernel calls \texttt{malloc} or \texttt{free}.
\revision{This heap tracking provides the ability to check for dynamic allocation errors (e.g., use-after-free and double-free bugs), such as in Valgrind\cite{valgrind}.}

Landslide also maintains a set of shared memory accesses made since the last decision point, for use with Partial Order Reduction (Section~\ref{sec:por-independence}). Whenever the guest kernel accesses memory in its heap or in its global data regions, Landslide adds the address of the access to the set, with a flag indicating whether it was a read or a write. (If the address was already present, and the recorded access was a read and the current access is a write, we upgrade the recorded access to a write. Otherwise if the address was already present, we do nothing.)

Landslide ignores shared memory accesses from the kernel's dynamic allocator itself, and it also ignores shared memory accesses from the components of the kernel's scheduler which run every transition (Section~\ref{sec:por-independence}).

When Landslide reaches a decision point, this accumulated set is copied in the decision tree (Section~\ref{sec:components-explore}), and reset to empty before continuing execution.

\subsection{The Arbiter}
\label{sec:components-arbiter}

The arbiter identifies points during execution that should count as decision points. The selection is mainly controlled by the user, during the annotation and configuration process. In addition, the arbiter also automatically identifies {\em voluntary reschedules}, which comprise the ``minimal necessary set'' of decision points.

In Landslide, it is always necessary to identify a decision point during a voluntary reschedule, to maintain the invariant that each transition between decision points is comprised of the execution of only one thread.
\revision{In future work, we may also investigate the implications of allowing some transitions to consist of multiple threads' executions, which would have applications for heuristic state space reduction.}

\revision{
Because voluntary reschedules happen of the guest kernel's own volition, when a thread begins one, it will necessarily not have been preempted by the timer handler. Hence, we identify voluntary reschedules when a thread enters the context switcher without having entered the timer handler.\footnote{
Landslide actually identifies voluntary reschedules at the point when the reschedule ends, so the {\em previous} thread would not have entered the timer handler, and the current thread may or may not have.
We do this to avoid the complexity of dealing with the possibility that the guest kernel would have interrupts disabled at the beginning of the reschedule and not re-enable them until after the thread switch.}
}

%\begin{itemize}
%	\item The previously-running thread entered the context switcher without first entering the timer interrupt handler,
%\end{itemize}
%
%and {\em any one} of the following:
%\begin{itemize}
%	\item The currently-running thread was handling a timer interrupt, and exits the {\em timer interrupt handler}.
%	\item The currently-running thread was not handling a timer interrupt, and exits the {\em context switcher}.
%	\item The currently-running thread was {\em just forked} (i.e, it has never run before), and hence may never exit either the timer handler or context switcher.
%\end{itemize}

In future work (Section~\ref{sec:future-new}), the arbiter may also automatically identify extra decision points, such as conflicting shared memory accesses.

\subsection{The Explorer}
\label{sec:components-explore}

The explorer maintains a representation of the current branch of the decision tree. It is responsible for checkpointing the state of both Landslide and the guest kernel at each decision point, deciding at the end of the test which branch of the tree to execute next (i.e., selecting which decision point should have been decided differently), and backtracking to appropriate points in the test's execution.

At each decision point, Landslide creates a new node in the decision tree. It stores the TID of the thread that was chosen, the state of the scheduler (which threads are on the runqueue), the state of the heap and the accumulated set of shared memory accesses, and a stack trace of the thread that was running.

\subsection{Test Lifecycle}
\label{sec:components-test}

\begin{figure}[h]
	\begin{center}
		\includegraphics[width=0.8\textwidth]{teststate.png}
	\end{center}
	\caption{The test lifecycle state machine.}
	\label{fig:teststate}
\end{figure}
Landslide knows the state of the test case through a simple state machine, which is updated with information about the number of threads currently on the system, the number of threads on the runqueue, the state of the idle thread (if it exists), and whether or not the shell is waiting for keyboard input.

The state machine relies on certain information provided by the scheduler (Section~\ref{sec:components-sched}).

\begin{itemize}
	\item {\bf Is the shell waiting for keyboard input?} In general, the shell only waits for keyboard input when the kernel is ready for the user (or Landslide) to cause a test to run, or after the test has finished and a new one could be run.
		However, the shell may sometimes be waiting for keyboard input while other threads have some work to do (for example, the init process might not yet have reaped a reparented child process), and this work may be affected by race conditions during the test, so we need to include it in the test as well. For these cases an additional check is needed.
	\item {\bf Is the kernel idling?} This indicates whether any work is left to be done during bootup or during teardown.
	\item {\bf How many threads currently exist on the system?} This indicates whether the test has begun running.
\end{itemize}

The states in the test lifecycle, and the rules for transitioning between them, are as follows.

\begin{enumerate}
	\item {\bf Bootup.} Initial state. When the kernel is idling and the shell is waiting for keyboard input, advance state.
	\item {\bf Keyboard input.} Record the current thread count (the ``start population''). Generate keystrokes to cause the desired test case to run.
		When the current thread count is greater than the start population, advance state.
	\item {\bf Test running.} Begin constructing the decision tree here (since it is pointless to backtrack to before the test begins). When the current thread count equals the start population, advance state.
	\item {\bf Test cleanup.} The test processes have exited, but some housekeeping may yet remain. When the the kernel is idling and the shell is waiting for keyboard input, advance state.
	\item {\bf Test ended.} Invoke the explorer (Section~\ref{sec:components-explore}) to decide whom to rewind, and backtrack (to step 3).
\end{enumerate}

Figure~\ref{fig:teststate} depicts the states and transitions visually.

\section{Identifying Bugs}
\label{sec:techniques-bugs}

In order to identify when the guest kernel has done something incorrect, Landslide performs several different types of checks, some accurate but noncomprehensive, and some heuristic-based.

\subsection{Definite Bug-Detection Conditions}

\begin{enumerate}
	\item {\bf Kernel panic bugs.} If the kernel invoked \texttt{panic}, it detected its own bug, and Landslide need do nothing but report it.
	\item {\bf Use-after-free bugs.} Whenever the kernel accesses memory in the heap (when not in the dynamic allocator itself), Landslide verifies that the address is within an allocated range. If not, Landslide proclaims the access to be illegal.\footnote{In the same way, Landslide also detects calls to \texttt{free} on blocks that were already freed or never allocated at all.}
	\item {\bf Deadlock bugs.} If Landslide finds no runnable threads on the runqueue,\footnote{Except for idle, if it exists.} or if it detects a cycle of threads blocked on each other, it declares that the kernel has deadlocked.
\suspend{enumerate}

\subsection{Probable Bug-Detection Conditions}

\revision{
\begin{figure}[ht]
	\centering
	\begin{tabular}{cc}
	\subfloat[Infinite loop without decision points. Disproportionately many instructions have been executed since the red state, so the kernel is probably stuck in a ``tight'' loop.]{
	\hspace{0.12\textwidth}
	\includegraphics[width=0.2\textwidth]{tightloop.png}
	\hspace{0.12\textwidth}
	} &
	\subfloat[Infinite loop around decision points. Disproportionately many decision points have been encountered on the red branch, so the kernel is probably stuck in a loop around certain decision points.]{
	\hspace{0.12\textwidth}
	\includegraphics[width=0.2\textwidth]{livelock.png}
	\hspace{0.12\textwidth}
	}
	\end{tabular}
	\caption{Landslide judges whether the kernel has gotten stuck by analysing the structure of the execution tree. The red branches indicate non-terminating thread interleavings, which Landslide would identify by comparing them with other branches in their respective trees.}
	\label{fig:infiniteloop}
\end{figure}
}

\resume{enumerate}
	\item {\bf Memory leak bugs.} Landslide records the state of the heap before the test case begins, and compares it to the state of the heap after the test case ends. If memory allocated during the test was not freed, Landslide assumes that it was leaked.
		(Some kernel designs may legitimately behave this way, so this bug-check may be disabled when testing such kernels.)\footnote{There is, of course, much room for improvement in this metric, but it is not part of the research contribution.}
	\item {\bf Infinite loop bugs.} Landslide judges whether the kernel has entered an infinite loop by comparing the current branch of the decision tree to past executions of the same test case.
	Figure~\ref{fig:infiniteloop} depicts such tree structures in abstract.
	\begin{itemize}
		\item {\bf Infinite loops without decision points.} While exploring the decision tree, Landslide computes the average number of instructions executed between two consecutive decision points. If at any point the current number of instructions executed since the most recent decision point exceeds this average times a constant factor (arbitrarily chosen to be 2000), Landslide assumes the kernel must have gotten stuck in an infinite loop.
		\item {\bf Infinite loops around decision points.} Landslide also computes the average number of decision points in each branch of the decision tree (the average ``branch depth''). If the depth of the current branch ever exceeds this average times a constant factor (arbitrarily chosen to be 20), Landslide assumes the kernel must have gotten stuck in an infinite loop.
	\end{itemize}
	\revision{In our experience, these heuristics have never falsely identified an infinite loop when the kernel was making real progress.
	However, both of these heuristics require a minimum number of branches to be explored before Landslide considers the already-explored tree structure to be statistically significant (arbitrarily chosen to be 20).
	As such, these heuristics occasionally fail to trigger on a real infinite loop in cases when the loop occurs before enough ``safe'' branches were explored. In the future we might improve this heuristic by scaling the associated cutoff factor depending on how many branches were explored: the fewer branches explored so far, the less reliable the comparison, and hence the higher the cutoff factor to be used.}
\end{enumerate}

\section{Partial-Order Reduction}
\label{sec:techniques-por}

We make use of Dynamic Partial-Order Reduction (DPOR), the state-space pruning algorithm presented in \cite{dpor}. DPOR requires two sets to be computed that describe the concurrency relationship between transitions: the happens-before relation, and the memory independence relation. Here we discuss the specifics of implementing these in our environment.

\subsection{Happens-Before Relation}

\revision{The happens-before relation expresses for each pair of transitions whether executing the first one is required to ``enable'' the second one. In order to establish the relation,} Landslide uses the state of the scheduler runqueues (Section~\ref{sec:components-sched}), which are snapshotted at every decision point (Section~\ref{sec:components-explore}).

Unfortunately, a thread's presence or absence on the scheduler runqueue does not necessarily correspond to whether it is {\em runnable} at any point. We identify three exceptions, as foreshadowed in Section~\ref{sec:challenges-design}:

\begin{itemize}
	\item {\bf Current thread not on runqueue.} If the currently-running thread is not stored on the runqueue, we identify it as runnable anyway (except in the special case of idle, as described below).
	\item {\bf Idle thread.} Some kernels may have an explicit idle thread and store it on the runqueue, with explicit code to skip over it if other threads are runnable. Hence, if the idle thread is on the runqueue (and/or the current thread), it is only runnable if no other threads are runnable.
	\item {\bf Yielding mutexes.} In kernels whose mutexes (or other synchronisation idioms) leave threads on the runqueue when they are notionally blocked, Landslide classifies such threads as not runnable.
\end{itemize}

Using this notion of runnability, the happens-before relation is established as follows. Let $X$ be a transition and $Y$ be a subsequent transition in a branch that was just explored, and $T_Y$ be the thread associated with $Y$. We say that $X$ {\em enables} $Y$ if $T_Y$ was not runnable immediately before $X$, but was runnable immediately after. Then, a transition $A$ happens-before a transition $B$ if $T_A$ is the same thread as $T_B$, if $A$ enables $B$, or if $A$ happens-before some transition $C$ that enables $B$.

\subsection{Memory Independence Relation}
\label{sec:por-independence}
The independence relation expresses which transitions do not read-and-write or write-and-write to the same shared memory addresses. Landslide computes this relation using the set of shared memory accesses described in Section~\ref{sec:components-memory}. We encountered two challenges pertaining to memory independence:

\begin{itemize}
	\item {\bf Always-accessed memory locations.}
		In kernel-space, every thread switch goes through common scheduler and context-switcher routines. These routines inevitably access scheduler data structures, such as the runqueue.\footnote{
		Another example is the timer tick counter: this global variable is not necessarily accessed every transition, because voluntary reschedules do not involve timer interrupts, but every involuntary preemption will cause a write to this counter.}
		If we include such memory accesses in the independence relation, it will result that all transitions conflict, and DPOR won't be able to achieve any reduction. To allow for state space reduction, we sacrifice our ability to find races involving these particular accesses by ignoring them. Currently we require the user to identify these locations as part of the instrumentation process, as described in Section~\ref{sec:using-config-landslide}.
	\item {\bf Freed memory poisoning.}
		A use-after-free bug happens whenever one thread accesses an address within an allocated block that another thread previously freed. Even if no other code makes a conflicting access to the same address after the \texttt{free}, accessing dynamically-allocated memory after it has been freed is illegal no matter what.

		Hence, even if the second thread that freed the block never accesses the particular address that the first thread used, the second thread's \texttt{free} still logically conflicts with the first thread's access. Landslide addresses this by treating every call to \texttt{free} as a {\em write} access to every address within the freed block when computing shared memory conflicts.
\end{itemize}

\subsection{Soundness}

\revision{
Because we recommend using Landslide with coarse-grained decision sets, as opposed to mandating a decision point between every pair of shared memory accesses, it is possible for Landslide to overlook race conditions that require finer-grained interleavings to expose. In this way, Landslide's search is not ``sound'', but instead false-negative-oriented.

We do, however, claim that Landslide's search is sound in a different way: if the provided decision points are sufficient to express an interleaving that would expose a race, then Landslide will find it. That is, if Landslide is exploring an execution tree with bugs in some branches, it will eventually find a bug.

Dynamic Partial Order Reduction works by identifying ``evil ancestors'' for each transition in a branch after executing that branch. Intuitively, a transition's evil ancestor is another transition such that executing the test with the order of those two transitions reversed could cause different concurrent behaviour to arise.}
\cite{dpor} and \cite{dbug-retreat} present a partial order reduction algorithm in which only the first evil ancestor of each transition need be considered when identifying which alternate interleavings need to be explored.

\begin{figure}[h]
	\centering
	\begin{tabular}{cc}
	\subfloat[Race-free interleaving.]{
		\footnotesize
		\begin{tabular}{l|c|c|c|}
			\cline{2-4}
			& {\bf Thread 1} & {\bf Thread 2} & {\bf Thread 3} \\
			\cline{2-4}
			$A$ & \texttt{x=42;} & & \\
			$B$ & & \texttt{y=31337;} & \\
			\multirow{2}{*}{$C_1$} & & & \texttt{if (y==31337)} \\
			& & & \texttt{assert(x==42);} \\
			\cline{2-4}
		\end{tabular}
	} &
	\subfloat[Alternate interleaving explored with DPOR.]{
		\footnotesize
		\begin{tabular}{l|c|c|c|}
			\cline{2-4}
			& {\bf Thread 1} & {\bf Thread 2} & {\bf Thread 3} \\
			\cline{2-4}
			$A$ & \texttt{x=42;} & & \\
			$C_2$ & & & \texttt{if (y==31337)} \\
			$B$ & & \texttt{y=31337;} & \\
			\cline{2-4}
		\end{tabular}
	} \\
	\multicolumn{2}{c}{
	\subfloat[Buggy interleaving.]{
		\footnotesize
		\begin{tabular}{l|c|c|c|}
			\cline{2-4}
			& {\bf Thread 1} & {\bf Thread 2} & {\bf Thread 3} \\
			\cline{2-4}
			$B$ & & \texttt{y=31337;} & \\
			\multirow{2}{*}{$C_3$} & & & \texttt{if (y==31337)} \\
			& & & \texttt{assert(x==42);} \\
			$A$ & \texttt{x=42;} & & \\
			\cline{2-4}
		\end{tabular}
	}}
	\end{tabular}
	\caption{This sample code defeats a DPOR implementation that uses only the first evil ancestor if a decision point is not defined between the two lines of thread 3's code. The first interleaving is explored first; DPOR identifies transition $B$ as the first evil ancestor of $C_1$, and runs the second interleaving next, reordering them. In the second interleaving, $A$ is no longer a second evil ancestor of $C_2$, so the third interleaving is never explored, and the assertion failure is missed.}
	\label{fig:evil-ancestors}
\end{figure}

We found that implementing DPOR in this way violated our second notion of soundness, because it relies on the invariant that each transition contains only one inter-thread communication event (in kernel-space, this means a shared memory access).
\revision{However, a key point of Landslide's recommended usage pattern is to sacrifice maximally-fine-grained decision points for performance, and so multiple different shared memory conflicts may happen in each transition.} Figure~\ref{fig:evil-ancestors} demonstrates a counterexample to the soundness of identifying only one evil ancestor per transition given our usage pattern.

\revision{Instead, Landslide uses considers every evil ancestor of each transition when performing DPOR. This modification enables Landslide to avoid the problem presented in the figure, because it will identify both alternate interleavings after exploring the first one. We claim without certainty that this change restores the second notion of soundness to Landslide's search, and leave proving this property to future work.}

\section{Debugging Feedback}
\label{sec:techniques-feedback}

\begin{figure}[h]
	%\includegraphics[width=0.95\textwidth]{found_a_bug.png}
	\centering
	\colorbox{black}{\color{white}
	{\small
	\begin{tabular}{l}
	\texttt{\hilight{green}{[SCHEDULE]}~~~~~~thread 4 vanished} \\
	\texttt{\hilight{green}{[SCHEDULE]}~~~~~~switched threads 4 -> 3} \\
	\texttt{\hilight{brown}{[MEMORY]}~~~~~~~~\hilight{red}{USE AFTER FREE - read from 0x0015a8f0 at eip 0x00104209}} \\
	\texttt{\hilight{brown}{[MEMORY]}~~~~~~~~Heap contents:~\{...\}} \\
	\texttt{\hilight{brown}{[MEMORY]}~~~~~~~~[0x15a8f0 | 4136] was allocated by TID3 at (...)} \\
	\texttt{\hilight{brown}{[MEMORY]}~~~~~~~~~~~~~~~~~~~~~~~~~~~~~~and freed by TID4 at (...)} \\
	\texttt{\hilight{red}{[BUG!]}~~~~~~~~~~\hilight{red}{****~~~~~A bug was found!~~~~****}} \\
	\texttt{\hilight{red}{[BUG!]}~~~~~~~~~~\hilight{red}{**** Decision trace follows.~****}} \\
	\texttt{\hilight{red}{[BUG!]}~~~~~~~~~~\hilight{yellow}{1:~~1347079 instructions, old 3 new 4, current 4}} \\
	\texttt{\hilight{red}{[BUG!]}~~~~~~~~~~~~~~TID3 at 0x00105a10 in \hilight{cyan}{context\_switch},} \\
	\texttt{\hilight{red}{[BUG!]}~~~~~~~~~~~~~~~~~~~~~~0x001041f4 in \hilight{cyan}{thread\_fork},} \\
	\texttt{\hilight{red}{[BUG!]}~~~~~~~~~~~~~~~~~~~~~~0x0010362b in \hilight{cyan}{thread\_fork\_wrapper}} \\
	\texttt{\hilight{red}{[BUG!]}~~~~~~~~~~\hilight{yellow}{2:~~1350725 instructions, old 4 new 3, current 3}} \\
	\texttt{\hilight{red}{[BUG!]}~~~~~~~~~~~~~~TID4 at 0x00105a10 in \hilight{cyan}{context\_switch},} \\
	\texttt{\hilight{red}{[BUG!]}~~~~~~~~~~~~~~~~~~~~~~0x00104681 in \hilight{cyan}{yield},} \\
	\texttt{\hilight{red}{[BUG!]}~~~~~~~~~~~~~~~~~~~~~~0x00104570 in \hilight{cyan}{vanish},} \\
	\texttt{\hilight{red}{[BUG!]}~~~~~~~~~~~~~~~~~~~~~~0x00103708 in \hilight{cyan}{vanish\_wrapper}} \\
	\texttt{\hilight{red}{[BUG!]}~~~~~~~~~~Stack:~TID3 at 0x00104209 in \hilight{cyan}{thread\_fork},} \\
	\texttt{\hilight{red}{[BUG!]}~~~~~~~~~~~~~~~~~~~~~~~~~0x0010362b in \hilight{cyan}{thread\_fork\_wrapper}} \\
	\texttt{\hilight{red}{[BUG!]}~~~~~~~~~~Total decision points 24, total backtracks 5} \\
	\texttt{\hilight{red}{[BUG!]}~~~~~~~~~~Average instrs/decision 16155, average branch depth 5} \\
	\end{tabular}
	}
	}
	\caption{A sample decision trace (with extraneous text trimmed) that Landslide generated using the \texttt{double\_thread\_fork} test case (Section~\ref{sec:using-tests}).}
	\label{fig:found_a_bug}
\end{figure}

In its unique position of control over when the kernel gets preempted and which thread gets scheduled at each context switch, Landslide has the capability to provide the user with detailed information about a test case's execution.
When Landslide determines that a bug was found, it immediately aborts exploration of the decision tree, and prints a {\em decision trace}: a comprehensive report of the particular interleaving of thread transitions that caused the bug to appear. Figure~\ref{fig:found_a_bug} depicts a simple decision trace for a use-after-free bug.

The decision trace explains each preemption or voluntary reschedule in the interleaving: which thread used to be running, which thread was chosen to run instead, and the stack trace of the former thread at the point from which it was switched away.

Landslide also prints the stack trace of the currently-running thread at the point where the bug was found, and (optionally) drops the user into the Simics debugging prompt. Depending on the nature of the bug found, Landslide also provides more detailed information:

\begin{enumerate}
	\item If the kernel has panicked, Landslide prints the message used in the panic/assertion.
	\item If a use-after-free bug is found, Landslide prints information about the most-recently-freed chunk containing the accessed address: the stack trace and TID both for when it was allocated and freed.
	\item If deadlock is detected, Landslide prints the cycle of TIDs that are blocked on each other.\footnote{This currently only works when using the annotations for yielding mutexes, though it is not difficult to implement more generally.}
	\item If a memory leak is suspected, Landslide prints how many bytes bigger the heap is after the test ended than when the test began.\footnote{Future feature: Printing when-allocated stack traces for each suspicious heap chunk.}
	\item If an infinite loop is suspected, depending on which heuristic was triggered, Landslide prints either the number of instructions since the last decision point (and the previous average) or the current branch depth (and the previous average).
\end{enumerate}

Landslide can also provide other useful information, even in cases where it did not find bugs, to help with the user's process of configuring decision points. Instead of exploring alternative interleavings, it can stop execution after the first interleaving and print out the set of decision points that were identified, along with the shared memory conflicts and happens-before relations for pairs of transitions between the decision points.

%%%%%%%%%%%%%%%%%%%%%%%%%%%%%%%%%%%%%%%%%%%%%%%%%%%%%%%%%%%%%%%%%%%%%%%%%%%%%%%%
\chapter{Using Landslide}
%%%%%%%%%%%%%%%%%%%%%%%%%%%%%%%%%%%%%%%%%%%%%%%%%%%%%%%%%%%%%%%%%%%%%%%%%%%%%%%%
\label{sec:using}

In this chapter we discuss how to use Landslide with a kernel that meets 15-410's Pebbles specification. We list some design requirements that a Pebbles kernel must meet, document the instrumentation process and how to customise Landslide's search behaviour, and tell how to interpret Landslide's results.

In preparation for conducting the student user study (Section~\ref{sec:eval-studence}), we distributed a user guide roughly similar to this chapter.\footnote{
The user guide itself is available at \url{http://bblum.net/landslide-guide.pdf}, and at \url{http://www.contrib.andrew.cmu.edu/~bblum/landslide-guide.pdf}.}

%%%%%%%%%%%%%%%%%%%%%%%%%%%%%%%%%%%%%%%%%%%%%%%%%%%%%%%%%%%%%%%%%%%%%%%%%%%%%%%%
\section{Kernel Requirements}
\label{sec:using-requirements}

\subsection{Scheduler Functionality}
\label{sec:using-requirements-sched}

In order to cause desired preemptions, Landslide needs to assume that the core of the scheduler works. In short, this means that timer interrupts will trigger context switches, and that if a thread is ``runnable'', at any point where interrupts/preemption is enabled, a finite number of timer preemptions will eventually cause that thread to run.

For deadlock detection, if the kernel has mutexes which loop around a call to \texttt{yield} (or similar), they must be annotated as described in section~\ref{sec:using-annotations}. \revision{If mutexes explicitly deschedule blocking threads, no special annotations are needed, because the other annotations do the job.}

Landslide does not support kernels that spin-wait anywhere, whether in mutexes or otherwise, especially when waiting for keyboard input or in \texttt{sleep}. Relatedly, the kernel must never run the idle loop when not truly idle, because Landslide uses this as a way of telling when all threads are wedged and/or when the test is done running.

\subsection{Virtual Memory}

Landslide does not have much to do with the virtual memory system, but the kernel must direct-map most of kernel memory, including the heap, globals, and kernel thread stacks. Landslide needs this to read values out of memory and to identify memory conflicts.\footnote{
It would be simple to support more advanced virtual memory set-ups, in which the same virtual address has different physical memory mappings at different times, but this is not implemented.}

The kernel should also use LMM (the \texttt{malloc} package in the 15-410 base code) for dynamic allocation, \revision{to avoid having to manually instrument a custom allocator.}

\subsection{System Calls}

In order to run with Landslide, the kernel must be able to boot up to the shell prompt,\footnote{
In Pebbles, the bootup sequence is usually fairly simple, involving the init process spawning the shell.}
receive keyboard input to make the shell start a test case, and return to the shell prompt after the test case finishes. The kernel must have the following system calls implemented and working in at least a rudimentary way:
\texttt{fork}, \texttt{exec}, \texttt{vanish}, \texttt{wait}, \texttt{readline}.

The following system calls are exercised in some, but not all, of the test cases we distribute, so are helpful to have, but not necessary:
\texttt{thread\_fork}, \texttt{yield}.

Apart from the system calls which init and shell execute, the rest are irrelevant and do not need to be implemented to use Landslide on a Pebbles kernel.

%%%%%%%%%%%%%%%%%%%%%%%%%%%%%%%%%%%%%%%%%%%%%%%%%%%%%%%%%%%%%%%%%%%%%%%%%%%%%%%%
\section{Instrumenting Kernels with Landslide}
\label{sec:using-instrumenting}

Here we document the interface by which Landslide understands the execution of the guest kernel. Landslide tries to be as design-agnostic as possible, after assuming that the kernel implements the Pebbles specification, but it still needs to know about the implementation of certain abstractions within the kernel.

\subsection{In-Kernel Annotations}
\label{sec:using-annotations}

\newcommand\hilight[2]{\color{#1}#2\color{white}}
\begin{figure}[h]
	\centering
	\colorbox{black}{\color{white}
	{\small
	\begin{tabular}{l}
	\texttt{\hilight{green}{int}~thread\_fork()} \\
	\texttt{\{} \\
	\texttt{~~~~\hilight{green}{thread\_t}~*child = construct\_new\_thread();} \\
	\texttt{~~~~\hilight{cyan}{tell\_landslide\_forking}();} \\
	\texttt{~~~~add\_to\_runqueue(child);} \\
	\texttt{~~~~\hilight{cyan}{tell\_landslide\_decide}(); \hilight{purple}{/* Interrupt me here! */}} \\
	\texttt{~~~~\hilight{brown}{return}~child->tid;} \\
	\texttt{\}} \\
	\end{tabular}
	}
	}
	\caption{Example annotated code. In theory, the use of \texttt{tell\_landslide\_decide()} in the place where a preemption needs to occur is not necessary for Landslide to know to preempt then; it is shown for the sake of demonstration.}
	\label{fig:tell-landslide}
\end{figure}

We provide a set of functions that the kernel needs to call at certain points during its execution to communicate what interesting events are happening. Figure~\ref{fig:tell-landslide} shows some example code annotated for Landslide.

\subsubsection{Important Note about Runqueues}
\label{sec:using-runqueue}
Landslide uses two of these annotations, \texttt{on\_rq()} and \texttt{off\_rq()}, for tracking the state of the guest kernel's scheduler runqueue at each point during execution.
Note that this refers to the {\em actual runqueue data structure}, which does not necessarily correspond to the abstract ``set of all runnable threads'', depending on whether or not the kernel keeps the current thread on the runqueue or off of it.

If the kernel does not keep the currently-running thread on the runqueue, the user must also use \texttt{kern\_current\_extra\_runnable} (Section~\ref{sec:using-student-c}).

\subsubsection{Required Annotations}

\revision{Some of these annotations may need to be used in ``sensitive'' regions of scheduler code, where it may not necessarily be clear at what point during the operation the annotation should go. Landslide assumes the scheduler already protects these regions by disabling preemption, and so the exact position of the annotation does not matter as long as it is within the preemption-disabled range.}

\begin{itemize}
	\small
	\item \texttt{tell\_landslide\_thread\_switch(int new\_tid)}: Call this in the context switcher, when a new thread is switched to.
	\item \texttt{tell\_landslide\_sched\_init\_done()}: Call this when the scheduler is done being initialized. Until this is called, Landslide will ignore all other annotations.
	\item \texttt{tell\_landslide\_forking()}: Call this when a new thread is being forked.
		(Hint: call it twice, once in \texttt{fork} and once in \texttt{thread\_fork}.)

		Call it ``just before'' the action which makes the new thread runnable. It does not matter whether the kernel just adds it to the runqueue (in which case the next annotation called would be \texttt{thread\_on\_rq}) or begins running it immediately (in which case \texttt{thread\_switch}); Landslide handles both cases.
	\item \texttt{tell\_landslide\_vanishing()}: Call this when a thread is about to go away.
		Call it ``just before'' the relevant context switch; i.e., the one that should never return.
	\item \texttt{tell\_landslide\_sleeping()}: Call this when a thread is about to go to sleep (that's \texttt{sleep()} the system call).
		Call it ``just before'' the relevant context switch; i.e., the one that won't return for as long as the thread requested to sleep.
	\item \texttt{tell\_landslide\_thread\_on\_rq(int tid)}: Call this when a thread is about to be added to the runqueue.
		(Make sure this call is done within whatever protection is also used for the runqueue modification itself.)
	\item \texttt{tell\_landslide\_thread\_off\_rq(int tid)}: Call this when a thread is about to be removed from the runqueue. (Same protection clause as above applies.)
	\item Mutex annotations. These are only necessary if the kernel uses mutexes that leave blocked threads on the runqueue, rather than explicitly descheduling them.
	\begin{itemize}
		\item \texttt{tell\_landslide\_mutex\_locking(void *mutex\_addr)}: Call this when a thread is about to start locking a mutex. Landslide needs the mutex address for deadlock detection, to track which mutex each thread is blocked on.
		\item \texttt{tell\_landslide\_mutex\_locking\_done()}: Call this when a thread finishes locking a mutex (and now owns it).
		\item \texttt{tell\_landslide\_mutex\_unlocking(void *mutex\_addr)}: Call this when a thread is about to start unlocking a mutex.
		\item \texttt{tell\_landslide\_mutex\_unlocking\_done()}: Call this when a thread finishes unlocking a mutex (and no longer owns it).
		\item \texttt{tell\_landslide\_mutex\_blocking(int owner\_tid)}: Call this when a thread is contending on a locked mutex, and hence is no longer logically ``runnable'' yet still on the runqueue. Landslide needs to know who owns the mutex for deadlock detection.
	\end{itemize}
\end{itemize}

\subsubsection{Optional Annotations}
\begin{itemize}
	\small
	\item \texttt{tell\_landslide\_decide()}: Call this to define additional decision points. See section~\ref{sec:using-decision}.
	\item \texttt{assert()} or \texttt{panic()} -
		The more important invariants for which a kernel has \texttt{assert}s, the more likely Landslide is to find bugs that would trigger them; otherwise, even if they do happen, Landslide might never know (just like in conventional stress testing).
\end{itemize}

\subsection{Configuration File}
\label{sec:using-config-landslide}

There is also a config file the user needs to fill out, called \texttt{config.landslide}. Some of the fields are required information, and some are tweaks that change the way Landslide behaves. We omit discussion of the required fields, and explain the optimisations and behaviour tweaks that Landslide supports using this file in Section~\ref{sec:using-search}.

\subsection{Instrumenting Within Landslide}
\label{sec:using-student-c}

There are two functions that the user needs to implement in Landslide itself, in a file called \texttt{student.c}, to express certain kernel designs that might be more complicated than a simple true/false condition or integer.

\begin{itemize}
	\small
        \item \texttt{bool kern\_current\_extra\_runnable(conf\_object\_t *cpu)}: See Section~\ref{sec:using-runqueue}. This says whether the current thread is ``logically runnable'' despite not being on the runqueue itself (that is, \texttt{tell\_landslide\_on\_rq} won't have been called for it).
        \item \texttt{bool kern\_ready\_for\_timer\_interrupt(conf\_object\_t *cpu)}: This function can be used to express when the scheduler is ``locked'' in a way not involving disabled interrupts. Landslide will avoid trying to preempt the kernel whenever this is false.
\end{itemize}

\revision{Both of these functions are predicates on the guest kernel's execution state (usually on the contents of its scheduler's data structures). Example implementations are given in Figure~\ref{fig:student-c}.}

\begin{figure}[h]
	% FIXME: replace with an image
	\centering
	\colorbox{black}{\color{white}
	{\small
	\begin{tabular}{l}
	\texttt{\hilight{green}{bool}~kern\_current\_extra\_runnable(\hilight{green}{conf\_object\_t}~*cpu)} \\
	\texttt{\{} \\
	\texttt{~~~~\hilight{green}{int}~tcb = \hilight{cyan}{READ\_MEMORY}(cpu, GUEST\_CURRENT\_TCB\_POINTER);} \\
	\texttt{~~~~\hilight{green}{int}~state\_flag = \hilight{cyan}{READ\_MEMORY}(cpu, tcb + GUEST\_TCB\_STATE\_FLAG\_OFFSET);} \\
	\texttt{~~~~\hilight{brown}{return}~state\_flag == GUEST\_TCB\_RUNNABLE;} \\
	\texttt{\}} \\
	\texttt{\hilight{green}{bool}~kern\_ready\_for\_timer\_interrupt(\hilight{green}{conf\_object\_t}~*cpu)} \\
	\texttt{\{} \\
	\texttt{~~~~\hilight{green}{int}~x = \hilight{cyan}{READ\_MEMORY}(cpu, GUEST\_PREEMPTION\_FLAG);} \\
	\texttt{~~~~\hilight{brown}{return}~x == GUEST\_PREEMPTION\_ENABLED;} \\
	\texttt{\}} \\
	\end{tabular}
	}
	}
	\caption{Example implementations of the two functions in \texttt{student.c}.}
	\label{fig:student-c}
\end{figure}

%%%%%%%%%%%%%%%%%%%%%%%%%%%%%%%%%%%%%%%%%%%%%%%%%%%%%%%%%%%%%%%%%%%%%%%%%%%%%%%%
\section{Configuring Landslide's Behaviour}
\label{sec:using-customise}
\subsection{Decision Points}
\label{sec:using-decision}

Getting a good set of decision points is important to being able both to explore reasonably quickly and to produce meaningful interleavings that are likely to find bugs. After getting Landslide working with the default set of decision points (voluntary reschedules only), the next step is to add some more. We list here the general steps in this process.

\begin{itemize}
        \item Use \texttt{tell\_landslide\_decide()} to indicate decision points.\footnote{
	We recommend at the start of \texttt{mutex\_lock} and at the end of \texttt{mutex\_unlock}, though it is also important to use one's own intuition, depending on which system call(s) are being tested.}
        \item Use \texttt{sched\_func} and \texttt{ignore\_sym} to make Landslide ignore global memory accesses, to enable better pruning.
        \item Use \texttt{DECISION\_INFO\_ONLY} to examine the set of decision points, and figure out which ones are irrelevant.
        \item Use \texttt{within\_func} and \texttt{without\_func} to make Landslide only use relevant decision points.
\end{itemize}

\subsection{Search Parameters}
\label{sec:using-search}

Landslide's configuration file supports several options for changing the way it explores the decision tree.

\subsubsection{Optimizations}

These configuration options help Landslide identify potentially-conflicting shared memory accesses that it should ignore (Section~\ref{sec:por-independence}).

\begin{itemize}
	\small
	\item \texttt{sched\_func}: When functions that make up the kernel's scheduler are identified, Landslide knows to ignore shared memory accesses from them. Accesses to scheduler data structures happen during {\em every} transition, so ignoring these is necessary for achieving any reduction at all.
	\item \texttt{ignore\_sym}: The user may wish to ignore memory accesses from other global data structures as well. For example, ignoring a mutex which is locked very frequently, but irrelevant to the actual test case, will result in an independence relation that \revision{enables much better state space reduction without losing the ability to find bugs in the parts of the kernel the user is interested in testing}.
	\item \texttt{within\_function}: If the user configures decision points to happen in common code paths, such as \texttt{mutex\_lock}, it will likely generate more branching than needed.\footnote{
For example, if testing \texttt{vanish}, there's no need to branch on a \texttt{mutex\_lock} in \texttt{exec}.}
This configuration option allows the user to whitelist functions so that Landslide will only decide if the current thread is ``within'' one of those functions.
	\item \texttt{without\_function}: Similar to above, but a blacklist. These two can be used together; later invocations take precedence over earlier ones.
\end{itemize}

\subsubsection{Behaviour Tweaks}
\begin{itemize}
	\small
	\item \texttt{EXPLORE\_BACKWARDS}: In which order to explore the branches? Backwards means with more preemptions first; forwards means tending to let the current thread keep running more often. Backwards tends to find bugs more quickly, but produces longer decision traces. (Section~\ref{sec:future-backwards})
	\item \texttt{DECISION\_INFO\_ONLY}: This option makes Landslide stop after one branch, and output the list of decision points it identified using the current config. Useful to see all decision points, and tweak settings to get rid of frivolous ones.
\end{itemize}

%%%%%%%%%%%%%%%%%%%%%%%%%%%%%%%%%%%%%%%%%%%%%%%%%%%%%%%%%%%%%%%%%%%%%%%%%%%%%%%%
\section{Test Cases}
\label{sec:using-tests}

We ship Landslide with a suite of small test cases designed to expose several common races that students frequently encounter during 15-410 Project 3. We list their code in Appendix~\ref{sec:test-code}.

\begin{itemize}
        \item \texttt{vanish\_vanish}: Tests when a parent and child process \texttt{vanish()} simultaneously.
        \item \texttt{double\_wait}: Tests interactions of multiple waiters on a single child.
        \item \texttt{double\_thread\_fork}: Tests for interactions of multiple threads in one process vanishing.
        \item \texttt{yield\_vanish}: Tests for interactions between \texttt{yield()} and \texttt{vanish()}.
\end{itemize}

\subsection{Landslide-Friendly Test Characteristics}
\label{sec:using-landslide-friendly-tests}

It is important to note why all of these test cases are ``Landslide-friendly'': they all perform very little work on a single run, enabling Landslide to completely explore the state-space. They also run several, but not too many, threads at once, producing potentially interesting interleavings.

%%%%%%%%%%%%%%%%%%%%%%%%%%%%%%%%%%%%%%%%%%%%%%%%%%%%%%%%%%%%%%%%%%%%%%%%%%%%%%%%
\section{Evaluation}
%%%%%%%%%%%%%%%%%%%%%%%%%%%%%%%%%%%%%%%%%%%%%%%%%%%%%%%%%%%%%%%%%%%%%%%%%%%%%%%%
% 410 - talk about current state/methods (for students and for TAS)
% test suite presented
% process of instrumenting

\subsection{User Experience}

% TODO

%%%%%%%%%%%%%%%%%%%%%%%%%%%%%%%%%%%%%%%%%%%%%%%%%%%%%%%%%%%%%%%%%%%%%%%%%%%%%%%%
\subsection{Bug Case Studies}
\label{sec:eval-casestudy}

\subsubsection{Bug Descriptions}

% TODO: intro text goes here

\subsubsection{Numbers}

\newcommand\bugnum[2]{\textcolor{BrickRed}{{\bf #1} {\small \em (#2)}}}
\newcommand\nobugnum[2]{\textcolor{Blue}{{\em #1} {\small \em (#2)}}}

\begin{figure*}[t!]
	\begin{center}
	\begin{tabular}{|l||c|c|c|c||c|}
		\hline
		\multirow{2}{*}{\bf Kernel and test case} & \multicolumn{4}{c||}{\bf Custom decision points} & \multirow{2}{*}{\bf Stress test} \\
		\cline{2-5}
		& \bf default & \bf lock & \bf unlock & \bf both & \\
		\hline\hline
		POBBLES vanish/vanish (a) & \nobugnum{31.8}{0.6} & \bugnum{57.1}{1.7} & & & \\
		\hline
		POBBLES vanish/vanish (b) & \nobugnum{32}{0.4} & \bugnum{51.5}{2.1} & & & \\
		\hline
		POBBLES wait/wait & \bugnum{23.3}{0.7} & \bugnum{27.9}{0.8} & \bugnum{27.9}{2.1} & \bugnum{41.6}{1.4} & \\
		\hline
		POBBLES thread\_fork/vanish & \bugnum{22}{0.6} & \bugnum{37.4}{1.1} & \bugnum{27.6}{0.5} & \bugnum{72}{2.6} & \\
		\hline
		LudicrOS vanish/vanish & \nobugnum{13.2}{0.2} & \bugnum{13.7}{0.7} & \bugnum{34.6}{1.1} & \bugnum{17.1}{0.3} & \\
		\hline
		LudicrOS yield/vanish & \nobugnum{12.3}{0.3} & \bugnum{11.4}{0.4} & \nobugnum{27.4}{0.8} & \bugnum{11.7}{0.4} & \\
		\hline
	\end{tabular}
	\end{center}
	\caption{Comparison of time taken (in seconds) to find bugs using Landslide and using conventional stress testing.
	Landslide's times are given for several different sets of decision points: the default set, consisting only of voluntary reschedules (Section~\ref{sec:components-arbiter}); and using custom decision points in addition to the default set: calls to \texttt{mutex\_lock}, calls to \texttt{mutex\_unlock}, and both.
	All numbers represent the average elapsed time in seconds from 5 trials, and the standard deviation is shown in parentheses. ``\nobugnum{$X$}{$s$}'' indicates the bug was not found with a given test setup (and instead, Landslide took $X$ time to explore the entire decision tree). ``$\infty$'' indicates the bug was not found with stress testing. % TODO: after how long? what timeout?
	}
	\label{fig:numbers}
\end{figure*}

% TODO: have a graph

Table~\ref{fig:numbers} shows the time it takes to find each of these bugs using Landslide, configured with several different sets of decision points, and using conventional stress testing. The experimental set-up is as follows:

\begin{itemize}
	\item All Landslide trial times include the Simics start-up and kernel boot-up time (time between issuing the command and the test case beginning to run), roughly 15 seconds for POBBLES and 10 seconds for LudicrOS.
	\item All Landslide trials were run on the Gates-Hillman cluster machines (2.6 GHz Xeon).
	% TODO: say specs of crash machine
	\item All Landslide trials were run with ``backwards exploration'' enabled (Section~\ref{sec:using-search}).
	\item All trials were also run with Landslide configured to pay attention to only the relevant system calls (using \texttt{within\_function}; Section~\ref{sec:using-decision}).

	We believe it is reasonable to test for these bugs in this way - using the minimal set of system calls to be paid attention to as necessary to find the bug - because it follows the recommended workflow of using Landslide, which is to start with what the user judges to be the ``smallest relevant set'' of decision points. The configuration using \texttt{within\_function} was as follows.
	\begin{itemize}
		\item POBBLES vanish/vanish(a): \texttt{within\_function vanish}
		\item POBBLES vanish/vanish(b): \texttt{within\_function vanish}
		\item POBBLES wait/wait: \texttt{within\_function wait}
		\item POBBLES thread\_fork/vanish: \texttt{within\_function thread\_fork} and \texttt{within\_function vanish}
		\item LudicrOS vanish/vanish: \texttt{within\_function vanish}
		\item LudicrOS yield/vanish: \texttt{within\_function yield} and \texttt{within\_function vanish}
	\end{itemize}
\end{itemize}

%%%%

\newcommand\bugtree[1]{\textcolor{BrickRed}{\bf #1}}
\newcommand\nobugtree[1]{\textcolor{Blue}{\em #1}}
\begin{figure*}[t!]
	\begin{center}
	\begin{tabular}{|l|l||c|c|c|c|}
		\hline
		\multirow{2}{*}{\bf Kernel and test case} & \multirow{2}{*}{\bf Property of tree} & \multicolumn{4}{|c|}{\bf Custom decision points} \\
		\cline{3-6}
		& & \bf default & \bf lock & \bf unlock & \bf both \\
		\hline\hline
		\multirow{4}{*}{POBBLES vanish/vanish (a)} & Decision points & \nobugtree{56} & \bugtree{1296} & & \\
		& Total backtracks   & \nobugtree{16} & \bugtree{376} & & \\
		& Average branch depth & \nobugtree{5} & \bugtree{19} & & \\
		\hline
		\multirow{4}{*}{POBBLES vanish/vanish (b)} & Decision points & \nobugtree{56} & \bugtree{1295} & \nobugtree{382071} & \\
		& Total backtracks   & \nobugtree{16} & \bugtree{376} & \nobugtree{112706} & \\
		& Average branch depth & \nobugtree{5} & \bugtree{17} & \nobugtree{16} & \\
		\hline
		\multirow{4}{*}{POBBLES wait/wait} & Decision points & \bugtree{23} & \bugtree{102} & \bugtree{74} & \bugtree{378} \\
		& Total backtracks   & \bugtree{4} & \bugtree{17} & \bugtree{12} & \bugtree{56} \\
		& Average branch depth & \bugtree{6} & \bugtree{10} & \bugtree{9} & \bugtree{14} \\
		\hline
		\multirow{4}{*}{POBBLES thread\_fork/vanish} & Decision points & \bugtree{24} & \bugtree{394} & \bugtree{273} & \bugtree{2269} \\
		& Total backtracks   & \bugtree{5} & \bugtree{70} & \bugtree{56} & \bugtree{410} \\
		& Average branch depth & \bugtree{5} & \bugtree{16} & \bugtree{14} & \bugtree{23} \\
		\hline
		\multirow{4}{*}{LudicrOS vanish/vanish} & Decision points & \nobugtree{10} & \bugtree{16} & \bugtree{141} & \bugtree{42} \\
		& Total backtracks   & \nobugtree{2} & \bugtree{3} & \bugtree{48} & \bugtree{10} \\
		& Average branch depth & \nobugtree{2} & \bugtree{7} & \bugtree{9} & \bugtree{14} \\
		\hline
		\multirow{4}{*}{LudicrOS yield/vanish} & Decision points & \nobugtree{8} & \bugtree{5} & \nobugtree{149} & \bugtree{7} \\
		& Total backtracks   & \nobugtree{1} & \bugtree{0} & \nobugtree{43} & \bugtree{0} \\
		& Average branch depth & \nobugtree{2} & \bugtree{0} & \nobugtree{9} & \bugtree{0} \\
		\hline
	\end{tabular}
	\end{center}
	\caption{Information about the decision trees explored when finding bugs. As in the previous table, each test case was run with the four different sets of decision points. ``\nobugtree{X}'' means the tree was completely explored because Landslide did not find a bug in that configuration. ``\bugtree{X}'' reflects the portion of the tree that was explored before a bug was found.}
	\label{fig:trees}
\end{figure*}

Table~\ref{fig:trees} presents more detailed information about the decision trees that Landslide explored when finding these bugs.
For each set of decision points on each bug, we give the total number of decision points in the tree, the total number of backtracks (i.e. branches explored before the bug was found), and the average branch depth (i.e. number of decision points in each branch).

%%%%%%%%%%%%%%%%%%%%%%%%%%%%%%%%%%%%%%%%%%%%%%%%%%%%%%%%%%%%%%%%%%%%%%%%%%%%%%%%
\subsection{Summary of Bugs Found}

% TODO

%%%%%%%%%%%%%%%%%%%%%%%%%%%%%%%%%%%%%%%%%%%%%%%%%%%%%%%%%%%%%%%%%%%%%%%%%%%%%%%%
\subsection{Discussion}

\subsubsection{Invariants}

While evaluating Landslide on these bugs, we determined two invariants that should hold for multiple explorations on the same test case.

\begin{enumerate}
	\item {\bf Ordering invariant}: For a given set of decision points, exploring the tree in multiple different orders (``forwards''/``backwards'') should produce the same result in terms of whether a bug was found or not. The bugs found may be different, and the number of branches explored may be different, but it should never be that one ordering finds a bug while another ordering of the same tree finds no bug.
	\item {\bf Superset invariant}: For a given set of decision points, if an exploration of the resulting tree finds a bug, using a superset of that set of decision points should also find a bug. This is because the first tree will be a sub-tree of the second, as shown by never preempting at a decision point that appears in the second set but not the first.
\end{enumerate}

In short, even though Landslide may give false negatives from using imperfect sets of decision points, the exploration itself must be sound (i.e. not missing any bugs that exist in the resulting tree).\footnote{
When running LudicrOS yield/vanish with decision points on \texttt{mutex\_unlock} but not on \texttt{mutex\_lock}, we found that the ordering invariant failed - ``backwards'' exploration found no bug, while ``forwards'' exploration did. We attribute this to a bug in Landslide itself, and present the results for the backwards exploration as usual, in which Landslide found no bug.}

\subsubsection{Recommended Testing Strategies}

% TODO

% vim: ft=tex

%%%%%%%%%%%%%%%%%%%%%%%%%%%%%%%%%%%%%%%%%%%%%%%%%%%%%%%%%%%%%%%%%%%%%%%%%%%%%%%%
\section{Future Work}
%%%%%%%%%%%%%%%%%%%%%%%%%%%%%%%%%%%%%%%%%%%%%%%%%%%%%%%%%%%%%%%%%%%%%%%%%%%%%%%%
\label{sec:future}

% Still other efforts present testing techniques orthogonal to systematic exploration, which could augment its effectiveness if combined in one tool.

\subsection{Interface Improvements}
\label{sec:future-interface}
% make user able to replay the choices
% better decision trace representation
% in general, feedback from the tests

\subsection{New Techniques}
\label{sec:future-new}
% new techniques for landslide:
% - hybrid with data-race/lockset/static-analysis; automated shm choices
% - parallelism
% - ICB as in chess
% - hybrid with symbolic execution
% - choice trace minimization

symbolic execution\cite{klee,dawson}

\subsection{Education}
\label{sec:future-education}
% teaching tool, study thinking patterns
% 410 stuff

\subsection{Production Kernels}
\label{sec:future-linux}
% linux, device drivers
% embedded microcontroller operating systems, simulated

\subsection{Performance}
\label{sec:future-perf}
% VM instead of simics
% - how to interpose? single-step, or unmap heap, and/or annotate (for hooks) with hypercalls
% - how to time travel? instead of time travel, snapshot once and replay

\subsection{Long-Running Testing Approaches}
\label{sec:future-shaping}
% - garth's shaping

\subsection{Theoretical Oddities}
\label{sec:future-theory}
% theory things to study:
\subsubsection{``Backwards'' Exploration}
\label{sec:future-backwards}
% - study the implications of backwards vs forward exploration
\subsubsection{Exploration Tree Structure}
% - nadim's backtrack depth

%%%%%%%%%%%%%%%%%%%%%%%%%%%%%%%%%%%%%%%%%%%%%%%%%%%%%%%%%%%%%%%%%%%%%%%%%%%%%%%%
\chapter{Conclusion}
%%%%%%%%%%%%%%%%%%%%%%%%%%%%%%%%%%%%%%%%%%%%%%%%%%%%%%%%%%%%%%%%%%%%%%%%%%%%%%%%

\newcommand{\s}{\fontfamily{jkpvos}\selectfont{}s}

% TODO write this for real
And lo, the Author did{\s}t pre{\s}ent Land{\s}lide to yon re{\s}earch Community, and to the {\s}tudents of XV-CDX as well. Verily, the Kernel developers young and old found Bugges of Concurren{\s}y in their code. The {\s}oftware thenceforth worked as Intended, and there was much rejoicing.


% TODO: read presubmit.txt and check everything off

\bibliography{citations}{}
\bibliographystyle{alpha}

\appendix
\chapter{Code for Provided Test Cases}
\label{sec:test-code}

The test cases that use the \texttt{thread\_fork} system call are written in assembly, because thread creation can only safely be done from C using a user-space thread library. However, using a user-space thread library for thread creation might forcibly serialise certain operations that would expose kernel bugs only if concurrent, and also necessitates many more system calls, which dramatically increases the size of the decision tree.

\section{\texttt{vanish\_vanish.c}}
%\lstinputlisting[language=C]{vanish_vanish.c}
{\small
\begin{verbatim}
#include <syscall.h>

void main()
{
    fork();
    vanish(); /* parent and child process exit simultaneously */
}
\end{verbatim}
}
\section{\texttt{double\_wait.S}}
%\lstinputlisting[language={[x86masm]Assembler}]{double_wait.S}
{\small
\begin{verbatim}
#include <syscall_int.h>

.global main
main:
    int $FORK_INT
    cmp $0x0,%eax
    jne parent
    int $VANISH_INT
parent:
    int $THREAD_FORK_INT
    mov $0x0,%esi
    int $WAIT_INT        # two parent threads do wait(NULL);
    int $VANISH_INT
\end{verbatim}
}
\section{\texttt{yield\_vanish.S}}
%\lstinputlisting[language={[x86masm]Assembler}]{yield_vanish.S}
{\small
\begin{verbatim}
#include <syscall_int.h>

.global main
main:
    int $THREAD_FORK_INT
    cmp $0x0,%eax
    jne parent
    int $VANISH_INT
parent:
    mov %eax,%esi
    int $YIELD_INT       # yield(child_tid);
    int $VANISH_INT
\end{verbatim}
}
\section{\texttt{double\_thread\_fork.S}}
%\lstinputlisting[language={[x86masm]Assembler}]{double_thread_fork.S}
{\small
\begin{verbatim}
#include <syscall_int.h>

.global main
main:
    int $THREAD_FORK_INT # fork 1st thread
    cmp $0x0,%eax
    je first_child
parent:
    int $VANISH_INT
first_child:
    int $THREAD_FORK_INT # fork 2nd thread
    int $VANISH_INT      # both threads vanish
\end{verbatim}
}


\end{document}
