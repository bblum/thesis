%%%%%%%%%%%%%%%%%%%%%%%%%%%%%%%%%%%%%%%%%%%%%%%%%%%%%%%%%%%%%%%%%%%%%%%%%%%%%%%%
\section{Design and Implementation}
%%%%%%%%%%%%%%%%%%%%%%%%%%%%%%%%%%%%%%%%%%%%%%%%%%%%%%%%%%%%%%%%%%%%%%%%%%%%%%%%
\subsection{Components of Landslide}

\subsubsection{Kernel Instrumentation}

\subsubsection{Scheduling}

\subsubsection{The Arbiter}

\subsubsection{The Decision Tree}

\subsection{On Simics}

Landslide is implemented as a module for Simics\cite{simics}

\subsection{Techniques}

\subsubsection{Thread Scheduling}

\subsubsection{Identifying Bugs}

% A healthy dose of snark. Perhaps cite the formally-proved haskell kernel here.
Without a formal specification of the internals of the guest kernel's implementation, it is of course impossible to identify both soundly and completely when a behaviour that constitutes a ``bug'' arises during a test case's execution.
In order to identify when the guest kernel has done something incorrect, Landslide performs several different types of checks, some accurate but noncomprehensive, and some heuristic-based.
We list the metrics here in order of reliability.
\begin{enumerate}
	\item Kernel panic bugs - 
	\item Use-after-free bugs - 
	\item Deadlock bugs - 
	\item Memory leak bugs - % this could fail if..
	\item Infinite loop bugs - f
\end{enumerate}
Landslide's bug reporting is false-negative oriented, meaning that it does not check for suspicious behaviours that might indicate underlying bugs, so it may report that it found no bugs even if some existed. If Landslide does report a bug, though, it is almost certainly correct (except in cases as listed above).

\subsubsection{Partial-Order Reduction}
\label{sec:techniques-por}
% talk about the all-ancestors vs induction thing
% talk about ignoring runqueue accesses, etc

\subsubsection{Selective Ignorance}

\subsection{Debugging Feedback}

In its unique position of control over when the kernel gets preempted and which thread gets scheduled at each context-switch, Landslide has the capability to provide the user with detailed information about a test case's execution.
When Landslide determines that a bug was found, it immediately aborts exploration of the decision tree, and prints a {\em decision trace} - a comprehensive report of the particular interleaving of thread transitions that caused the bug to appear.

The decision trace explains each preemption or voluntary reschedule in the interleaving: which thread used to be running, which thread was chosen to run instead, and the stack trace of the former thread at the point from which it was switched away.


% TODO: this sentence lies somewhat
Additional information, such as the number of instructions executed during each transition, and the shared memory accesses that conflicted with other transitions, can also be printed at the user's configuration options.
% Should it also print conflicting memory accesses? Configurable by the user?
% Make the user able to replay the choices.

% Discuss combining this with a choice trace minimisation tool. Reference future work.

% Have a section or subsection to talk about the tool-user relationship.
