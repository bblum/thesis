\chapter{Code for Provided Test Cases}
\label{sec:test-code}

The test cases that use the \texttt{thread\_fork} system call are written in assembly, because thread creation can only safely be done from C using a user-space thread library. However, using a user-space thread library for thread creation might forcibly serialise certain operations that would expose kernel bugs only if concurrent, and also necessitates many more system calls, which dramatically increases the size of the decision tree.

\section{\texttt{vanish\_vanish.c}}
%\lstinputlisting[language=C]{vanish_vanish.c}
{\small
\begin{verbatim}
#include <syscall.h>

void main()
{
    fork();
    vanish(); /* parent and child process exit simultaneously */
}
\end{verbatim}
}
\section{\texttt{double\_wait.S}}
%\lstinputlisting[language={[x86masm]Assembler}]{double_wait.S}
{\small
\begin{verbatim}
#include <syscall_int.h>

.global main
main:
    int $FORK_INT
    cmp $0x0,%eax
    jne parent
    int $VANISH_INT
parent:
    int $THREAD_FORK_INT
    mov $0x0,%esi
    int $WAIT_INT        # two parent threads do wait(NULL);
    int $VANISH_INT
\end{verbatim}
}
\section{\texttt{yield\_vanish.S}}
%\lstinputlisting[language={[x86masm]Assembler}]{yield_vanish.S}
{\small
\begin{verbatim}
#include <syscall_int.h>

.global main
main:
    int $THREAD_FORK_INT
    cmp $0x0,%eax
    jne parent
    int $VANISH_INT
parent:
    mov %eax,%esi
    int $YIELD_INT       # yield(child_tid);
    int $VANISH_INT
\end{verbatim}
}
\section{\texttt{double\_thread\_fork.S}}
%\lstinputlisting[language={[x86masm]Assembler}]{double_thread_fork.S}
{\small
\begin{verbatim}
#include <syscall_int.h>

.global main
main:
    int $THREAD_FORK_INT # fork 1st thread
    cmp $0x0,%eax
    je first_child
parent:
    int $VANISH_INT
first_child:
    int $THREAD_FORK_INT # fork 2nd thread
    int $VANISH_INT      # both threads vanish
\end{verbatim}
}
